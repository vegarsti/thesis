\chapter{Discussion and future work}
\label{sec:discussion}
In this thesis, we have looked at problems in survival data, and specifically first hitting time models.
While Cox regression is by far the most popular method used to estimate survival data models, there are shortcomings with the Cox model.
The first hitting time model is more flexible.
However there were no existing methods for estimating FHT models with high-dimensional data.
We have therefore discussed ways of estimating models that work well in such a setting.
In particular, we have discussed gradient boosting \citep{friedman2001}, both methods for estimating one parameter, and extensions to several parameters.

Our goal with the thesis work was therefore to combine FHT models and gradient boosting.
We have therefore developed an algorithm for fitting an FHT model with linear additive predictors.
The estimation algorithm works as follows.
It starts by initializing the additive predictors to intercepts, specifically the intercepts that maximize the log-likelihood of the training set.
We then perform iterations where we in each step include a regularized linear least squares function in \textit{one} of the covariates and \textit{one} of the parameters, namely the combination of covariate and parameter which leads to the largest increase in the log-likelihood function.
The algorithm was implemented from scratch as a package which we called \textit{FHTBoost}, and it is freely available for download at \verb|https://github.com/vegarsti/fhtboost|.
It can be installed directly in R by using a command in the DevTools R package \citep{devtools} called \verb|install_github|, namely \verb|install_github("vegarsti/fhtboost")|.

There are several interesting avenues for further work.
One is to apply FHTBoost on other real-life high-dimensional survival data sets.
This would allow for a broader view of its usefulness and predictive power on real-life problems.
Another hopeful avenue of further work would be to incorporate the FHT model into the existing ecosystem of gradient boosting packages in R.
This would mean including the parameterization which we use in this thesis into the package concerning censored GAMLSS models, \verb|gamboostLSS.cens| \citep{gamlsscens}.