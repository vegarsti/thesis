\chapter{Multivariate component-wise boosting on survival data}
In this chapter, we propose a component-wise boosting algorithm for fitting the inverse gaussian first hitting time model to survival data.

\section{Simulation of survival data}
We wish to simulate survival times $\ti,i=1,\ldots,N$ with censoring. We first draw survival times $\tilde{t}_i$ from some survival time distribution $f(\cdot)$. If this distribution has a closed form probability distribution function, we can draw from it directly. If not, we might use some an inverse sampling method, e.g. by drawing unit exponentials and using a corresponding transformation.

To censor the data, we draw censoring times $W_i\sim f(\cdot),i=1,\ldots,N$, from a more right-tailed distribution, meaning we want to get many, but not all, $W_i$'s to be larger than the $\tilde{t}_i$'s. We let the observed survival times then be $t_i=\min(\tilde{t}_i,W_i)$.
The corresponding observed indicator, $\di$, is then set equal to 1 if the actual survival time was observed, i.e., if $\ti<W_i$. We end up with a set of $N$ tuples $(t_i,\delta_i),i=1,\ldots,N$. Note that this scheme incorporates independent censoring: The censoring time is independent of the survival times.

\begin{algorithm}
\caption{Generating survival data from Inverse Gaussian FHT distribution}
\label{algo:FHT-sim}
\begin{enumerate}
    \item Given design matrices $\X$, $\Z$ and true parameter vectors $\bbeta$ and $\bgamma$.
    \item Link covariates and parameters using link functions
        \begin{align*}
            \ln y_0&=\bbeta^T\X \\
            \mu&=\bgamma^T\Z.
        \end{align*}
    \item Draw $N$ survival times $(t_i)_{i=1}^N$ from IG$(\mu,y_0)$.
    \item Draw a censoring time $W$ from some distribution which is independent of the data.
    \item Right censor data by choosing $\widetilde{t}_i=\min(t_i,W)$. The indicator on whether observation $i$ was observed or not is then $\delta_i=I(\widetilde{t}_i=t_i)$.
    \item The simulated data set is $(\widetilde{t}_i,\delta_i)_{i=1,\ldots,N}$.
\end{enumerate}
\end{algorithm}