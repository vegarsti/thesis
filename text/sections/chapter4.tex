\chapter{First hitting time boost}
In this chapter, we propose a component-wise boosting algorithm for fitting the inverse gaussian first hitting time model to survival data.

\section{Algorithm}
We apply the component-wise boosting algorithm \ref{algo:fhtboost} with loss function $\rho(\mu,\y0)=-\log\loss{y_0,\mu}$. We differentiate the loss function with respect to these two and get .... For more details on the derivation, see \ref{appendix}. \todo{Maybe use $b$ instead of $y_0$, to not get subscript chaos?}

\begin{algorithm}
\caption{FHT Boost with twodimensional loss function}
\label{algo:fhtboost}
\begin{enumerate}
    \item Initialize the $n$-dimensional vectors $\hat{y}_0^{[0]},\hat{\mu}^{[0]}$, with offset values, e.g. with $\hat{y}_0^{[0]}=\0,\ldots,\hat{\mu}^{[0]}=\0$. Alternatively, one can use the maximum likelihood estimates as offset values.
    \item For both components of the loss function, specify base learners, in particular, a component-wise base learner which can be used for each of the $p$ variables used in $\X$ corresponding to $y_0$ and the $d$ variables in $\Z$ corresponding to $\mu$. Like earlier, the base learner takes one input variable and has one output variable. Examples include least squares linear regression.
    \item Set $m=0$ and $\nu=0.1$.
    \item Increase $m$ by 1.
    \begin{enumerate}
        \item If $m>m_{\text{stop},y_0}$, proceed to step 4 e). If not, compute the negative partial derivative $-\frac{\partial\rho}{\partial y_0}$ and evaluate at $\hat{f}^{[m-1]}(X_i,Z_i)=\left(\hat{y}_0^{[m-1]}(X_i),\hat{\mu}^{[m-1]}(Z_i)\right)_{i=1,\ldots,n}$. This yields the negative gradient vector $U_{y_0}^{[m-1]}=\left(U_{i,y_0}^{[m-1]}\right)_{i=1,\ldots,n}:=\left(-\frac{\partial}{\partial y_0}\rho\left(Y_i,\hat{f}^{[m-1]}(X_i,Z_i)\right)\right)_{i=1,\ldots,n}$.
        \item Fit the negative gradient vector $U_{y_0}^{[m-1]}$ to each of the $p$ components of $\X$ separately (i.e. to each predictor variable) using the base learners specified in step 2. This yields $p$ vectors of predicted values, where each vector is an estimate of the negative gradient vector $U_{y_0}^{[m-1]}$.
        \item Select the component of $\X$ which best fits $U_{y_0}{[m-1]}$ according to $R^2$. Set $\hat{U}_{y_0}^{[m-1]}$ equal to the fitted values of the corresponding best model fitted in the previous step.
        \item Update $\hat{y}_0^{[m-1]}\gets\hat{y}_0^{[m-1]}+\nu\hat{U}_{y_0}^{[m-1]}$.
        \item If $m>m_{\text{stop},\mu}$, proceed to step 4 j). If not, compute the negative partial derivative $-\frac{\partial\rho}{\partial \mu}$ and evaluate at $\hat{f}^{[m-1]}(X_i,Z_i)=\left(\hat{y}_0^{[m-1]}(X_i),\hat{\mu}^{[m-1]}(Z_i)\right)_{i=1,\ldots,n}$. This yields the negative gradient vector $U_{\mu}^{[m-1]}=\left(U_{i,\mu}^{[m-1]}\right)_{i=1,\ldots,n}:=\left(-\frac{\partial}{\partial \mu}\rho\left(Y_i,\hat{f}^{[m-1]}(X_i,Z_i)\right)\right)_{i=1,\ldots,n}$.
        \item Fit the negative gradient vector $U_{\mu}^{[m-1]}$ to each of the $p$ components of $\Z$ separately (i.e. to each predictor variable) using the base learners specified in step 2. This yields $d$ vectors of predicted values, where each vector is an estimate of the negative gradient vector $U_{\mu}^{[m-1]}$.
        \item Select the component of $\Z$ which best fits $U_{\mu}{[m-1]}$ according to $R^2$. Set $\hat{U}_{\mu}^{[m-1]}$ equal to the fitted values of the corresponding best model fitted in the previous step.
        \item Update $\hat{\mu}^{[m-1]}\gets\hat{\mu}^{[m-1]}+\nu\hat{U}_{\mu}^{[m-1]}$.
        \item Update $\hat{f}^{[m]}\gets\hat{f}^{[m-1]}$.
        \item If $m>\max(m_{\text{stop},y_0},m_{\text{stop},\mu})$, go to step 5. If not, repeat step 4.
    \end{enumerate}
    \item Return $\hat{f}^{[m]}$.
\end{enumerate}
\end{algorithm}