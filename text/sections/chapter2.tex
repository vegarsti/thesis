\chapter{First hitting time regression models}
\label{sec:fht}

\section{First hitting time models}
An FHT model has two main components.
\begin{enumerate}
    \item A stochastic process $\{Y(t), t\in\setT,y\in\setY\}$, with $Y(0)=y_0$.
    \item A boundary, or threshold, $B\subset \setY$, where $y_0\notin\setB$
\end{enumerate}
The first hitting time is the first time the process reaches the boundary set. Formally, the FHT is a stochastic variable $S$, which is defined as
\[
    S = \inf\{t\colon Y(t)\in\setB\}
\]
Note that it is possible that $P(S<\infty)<1$.

This framework makes for a flexible model, due to many possible choices for the process and the boundary. We now look at a common choice of the process.

\subsection{Wiener process}
\label{wiener}
The Wiener process, also known as the standard Brownian motion process, is a process which is continuous in time and space, and has the properties (\cite[61]{caroni2017}) that
\begin{itemize}
    \item $Y(t)$ has independent increments, such that $Y(t_2)-Y(t_1)$ and $Y(t_4)-Y(t_3)$ are independent for any disjoint intervals, and
    \item for any interval $(t_1, t_2)$,
    \[
        Y(t_2)-Y(t-1)\sim N(\mu(t_2-t_1), \sigma^2(t_2-t_1)).
    \]
\end{itemize}
This process may both increase and decrease. However, if we want a monotonic restriction on the movement of the process, we may use a gamma process.

\subsection{Gamma process}
The gamma process is suitable for modelling something which we would require to be monotonic, typically a physical degradation, i.e. where the damage cannot mend itself. (\cite[59]{caroni2017})

Other choices of processes include Markov chain state models, the Bernoulli process, and the Ornstein-Uhlenbeck process.

\section{First hitting time regression based on underlying Wiener process}
The first hitting time of the Wiener process (section \ref{wiener}) process follows an inverse Gaussian distribution (derivation in \cite[23-29]{chhikara1988}):
\begin{equation}
\label{eq:fht-ig}
    f(t|y_0,\mu,\sigma^2)=\frac{y_0}{\sqrt{2\pi\sigma^2t^3}}\exp\left[-\frac{(y_0+\mu t)^2}{2\sigma^2t}\right]
\end{equation}
If $\mu$ is positive, $Y(t)\leq 0$ is not certain to occur. Note also that this parameterization is over-parameterized, because $Y$ has an arbitrary scale, so we can without loss of generality let $\sigma^2=1$.

While $\mu$ and $y_0$ have simple interpretations in terms of the underlying process, they do not in terms of the lifetime distribution. The mean lifetime is $\frac{y_0}{|\mu|}$, and the variance is $\frac{y_0}{|\mu|^3}$. (\cite[62]{caroni2017}.)

We may introduce effects from covariates by allowing $\mu$ and $y_0$ to depend on covariates $\x$. Suitable models are
\begin{align}
    \mu&=\bbeta^\T\u \\
    \ln y_0&=\bgamma^\T\v
\end{align}
where $\bbeta$ and $\bgamma$ are vectors of regression coefficients. $\u$ and $\v$ may be either completely different, partially different, or equal. In the latter case, we use $\bbeta^\T\x$ and $\bgamma^\T\x$ instead, for clarity.