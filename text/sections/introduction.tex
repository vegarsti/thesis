\chapter{Introduction and outline of the thesis}
\label{sec:intro}
The analysis of survival data is an important part of biomedical statistics.
Almost all modelling of survival data is done with the Cox regression model \citep{cox-model}.
A key underlying assumption of the Cox model is that the hazards of individuals are assumed to be proportional.
This assumption is not always valid.
Therefore there is a need for methods which are more flexible.

Enter first hitting time models \citep{leewhitmore2006}.
The first hitting time model framework is a rich and flexible modelling framework where the idea is to model the process underlying an individual before it experiences an event.
It is therefore a biologically feasible framework.
In this framework, we specify a stochastic process and a threshold, and when the process hits the boundary, the event is triggered.
For certain choices of processes, such as the Wiener process, there exist fully parameterized expressions which can be used in regression.

An important part of modern biomedical statistics is the ability to incorporate and use high-dimensional data, which is often genetic data.
Genetic data are nowadays widely available, and typical data sets include gene expressions from genes numbering in the tens of thousands.
Such data are an example of so-called high-dimensional data, because one can think of the genes from one individual as one point in a high-dimensional space where each gene spans one dimension.
Somewhat counterintuitively, virtually all points in high-dimensional space will be far apart.
This makes it difficult to generalize on the structure.
It also makes it very easy for statistical models to overfit, i.e., to explain the variation in the data in a way that is not really true, and that would not carry over to unseen data of a similar kind.
Furthermore, many classical statistical models are simply unable to use so many predictors, at least directly.
Specifically, a scenario where the number of covariates $p$, is much larger than the number of predictors, $n$, which is often referred to as the $p\gg n$ scenario.

There are models which extend Cox regression to such settings.

There do not exist such extensions for FHT models, and there has to our knowledge been no attempt at developing any methods for first hitting time regression in a high-dimensional setting.
Penalized maximum likelihood estimation methods are needed.
To estimate parameters in the Wiener FHT model in a high-dimensional setting, we cannot use the lasso \citep{lasso}, as it is used for one parameter.
In recent years, an algorithmical framework called gradient boosting has been very successful in high-dimensional ($p\gg n$) scenarios.
It is a method that originated at the end of the 20th century, in the field of machine learning.
\citet{friedman2001} later provided a statistical view.
Gradient boosting algorithms are iterative algorithms for performing regularized minimization of a loss function.
In statistics, this loss function is often the negative log likelihood of a distribution.
About 20 years later, gradient boosting is still very much an active field of research, and various methods exist for model fitting.
Most traditional gradient boosting algorithms are concerned with modelling one parameter.
In more recent years, however, gradient boosting methods for regression of parameters beyond the mean have also been introduced.

In this thesis, we develop such a multidimensional gradient boosting algorithm, to fit an FHT model which depends on two parameters.
Again, there has to our knowledge been no attempt at developing any methods for first hitting time regression in a high-dimensional setting.

In chapter 2, we give an overview of survival analysis and existing models, primarily Cox regression.
We then discuss first hitting time models and in particular a specific instance of such models, where a Wiener process is used.
Chapter 3 introduces boosting, and reviews existing methods for modelling the mean.
Furthermore it reviews methods for estimating several parameters using boosting.
In chapter 4, we derive such a multidimensional algorithm for fitting the FHT model discussed in chapter 2, which we call FHTBoost.
Chapter 5 gives an overview of evaluation measures that we then use in subsequent chapters.
In chapter 6 we perform a simulation study of FHTBoost, where we look at two high-dimensional survival data scenarios.
Chapter 7 looks at a survival data set consisting children diagnosed with neuroblastoma.
We estimate an FHT model and discuss the results, before we compare its predictive performance with that of a Cox regression performed on the same data.
We conclude the thesis with discussion and remarks on possible future work.