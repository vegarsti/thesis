\chapter{Introduction and outline of the thesis}
\label{sec:intro}
The analysis of survival data is an important part of biomedical statistics.
\todo[inline]{more about survival analysis!!}

Almost all practical modelling of survival data is done with the famous Cox regression model \citep{cox-model}.
A key underlying assumption of the Cox model is that individuals have proportional hazards at all times, i.e. the ratio of their hazards does not change over time.
This assumption is not always valid, especially in settings with variable selection.
There is a need for models which are more flexible, and one such family of models is first hitting time models (FHT) \citep{leewhitmore2006}.
Instead of modelling the hazard, the main idea of FHT models is to model the underlying (stochastic) process of an individual before an event occurs.
To model its life, so to speak.
The FHT model framework is a rich and flexible modelling framework, where we specify a stochastic process and an appropriate threshold, barrier, or boundary, depending on the association we wish to evoke.
There are many choices for this pair of properties.
When the process hits the barrier, or crosses the boundary, the event is triggered.
The time to event is therefore the time from the process started until it hit the barrier.
For certain choices of processes, there exist fully parameterized expressions which can be used in regression.
This is the case for the Wiener process with initial level and drift, which leads to a bivariate, or twodimensional, probability distribution.
In chapter 2, we discuss the survival analysis setting, Cox regression, and FHT models.
These are first discussed in general, before addressing in detail the specific case where we use a Wiener process.
This case leads to a lifetime which follows a version of the inverse Gaussian, which is a function of the initial level and the drift.
The typical regression setup is to use link functions to relate these to linear predictors which are functions of covariates.

An important part of modern biomedical statistics is the ability to incorporate and use high-dimensional data, which is often genetic data.
Genetic data are nowadays widely available, and typical data sets include gene expressions from genes numbering in the tens of thousands.
Such data are an example of so-called high-dimensional data, because one can think of the genes from one individual as one point in a high-dimensional space where each gene spans one dimension.
Somewhat counterintuitively, virtually all points in high-dimensional space will be far apart.
This makes it difficult to generalize on the structure.
It also makes it very easy for statistical models to overfit, i.e., to explain the variation in the data in a way that is not really true, and that would not carry over to unseen data of a similar kind.
Furthermore, many classical statistical models are simply unable to use so many predictors, at least directly.
Specifically, a scenario where the number of covariates $p$, is much larger than the number of predictors, $n$, which is often referred to as the $p\gg n$ scenario.

There are models which extend Cox regression to such settings, but it is not trivial how to extend these.
There do not, however, exist such extensions for FHT models, and there has to our knowledge been no attempt at developing methods for first hitting time regression in a high-dimensional setting.
Estimation methods which lead to regularized models are needed, and they especially need to have inherent variable selection.
One such method is the lasso \citep{lasso}, and another method is gradient boosting.
Gradient boosting is an algorithmical framework that has been very successful in high-dimensional ($p\gg n$) scenarios.
It is a method that originated at the end of the 20th century, in the field of machine learning.
\citet{friedman2001} later provided a statistical view.
Gradient boosting algorithms are iterative algorithms for performing regularized minimization of a loss function, which can very well be a negative log likelihood function.
Gradient boosting is still very much an active field of research, and various methods exist for model fitting.
Most traditional gradient boosting algorithms are concerned with modelling one parameter.
The Wiener FHT model we use is a bivariate model, and we can therefore not use the lasso, nor can we use the standard gradient boosting methods.
In more recent years, however, gradient boosting methods for regression of parameters beyond the mean have also been introduced.
In chapter 3, we discuss gradient boosting, up to an including methods for estimating multidimensional models.

This brings us to the main goal of this thesis.
In chapter 4, we develop a multidimensional, component-wise gradient boosting algorithm to fit the bivariate FHT model with Wiener processes, which we call \textit{FHTBoost}.
There has to our knowledge been no attempt at developing methods for first hitting time regression in a high-dimensional setting.
We implement this boosting algorithm entirely from scratch.
Using \textit{FHTBoost}, we can estimate the linear additive predictors of the FHT model.

Chapter 5 explains evaluation measures that we then use in subsequent chapters.
In chapter 6, we perform a simulation study where we verify that \textit{FHTBoost} manages to estimate the parameters of the FHT model.
We consider two scenarios, one with uncorrelated data, and a more realistic scenario with highly correlated data.
The penultimate chapter 7 looks at a survival data set consisting children diagnosed with neuroblastoma.
We estimate an FHT model on this data, and discuss the results.
We then compare its predictive performance with that of a boosted Cox regression.
Finally, we end the thesis with a conclusion and some remarks at future work.