\abstractintoc % Add abstract to Table of Contents  
\abstractnum   % Format abstract like a chapter
               % Remove if abstract should not be on its own page
\begin{abstract}
In this thesis, we consider models for survival data in a high-dimensional setting.
Specifically, we consider a typical setting in which both genomic data and clinical data are used to model the survival.
In a high-dimensional setting, models which can perform variable selection and shrinkage are needed.
Often, the Cox regression model is used.
It requires that hazard rates of individuals are proportional.
This is hard to verify in high dimensions, and it is hard to justify when doing variable selection.
Furthermore, when using both clinical and genomic data it is difficult to combine the data in a sensible way.
We therefore discuss first hitting time models, specifically with a Wiener health process.
Such models model the health process of each individual, based on an individual initial health level and an individual drift.
Within such a model, it is sensible to assign genomic data to the initial level, and the clinical data to the drift of the health process.
To estimate parameters in a first hitting time model with a Wiener health process, we implement a gradient boosting algorithm which we call \textit{FHTBoost}.
Boosting multivariate distributions is complex, and may be computationally intensive, but based on recent advances, we are able to make do with one tuning parameter.
We perform a simulation study where the implemented algorithm manages to achieve good variable selection and model fit.
Finally, \textit{FHTBoost} is applied to a survival data set where covariates consist of clinical and genetic data of children diagnosed with neuroblastoma.
As measured by Brier score, the \textit{FHTBoost} estimated model achieves a predictive power that is slightly worse, but comparable to a Cox model.
\end{abstract}