\chapter{Statistical boosting}
Boosting is one of the most promising methodological approaches for data analysis developed in the last two decades \citep{mayr14a}.
It has become a staple part of the statistical learning toolbox because it is a flexible tool for estimating interpretable statistical models.
Boosting, however, originated as a black box algorithm in the fields of computational learning theory and machine learning, not in statistics.

Computer scientists Michael Kearns and Leslie Valiant, who were working on computational learning theory, posed the following question: 
Could any weak learner be transformed to become a strong learner? \citep{kearnsvaliant}
A weak learner, sometimes also simple or base learner, is a learner that has a low performance.
For example, in the context of classification, a weak learner is one that performs only slightly better than random (uniform) chance.
In the binary classification setting, then, it would only perform slightly better than a coin flip.
Meanwhile, a strong learner should be able to perform in a near-perfect fashion, for example attaining high accuracy on a prediction task.
We will first give a summary of the history of boosting, starting with AdaBoost \citep{adaboost}, which proved that the answer to the original question above was yes.
For a complete overview of the history of boosting, see \citet{mayr14a, mayr14b, mayr17}.

\section{AdaBoost: From machine learning to statistical boosting}
The original AdaBoost, also called Discrete AdaBoost \citep{adaboost} is an iterative algorithm for constructing a binary classifier $F(\cdot)$.
It was the first \textit{adaptive} boosting algorithm, as it automatically adjusted its parameters to the data based on its perfomance.
In the binary classification problem, we are given a set of observations $(\x_i,y_i)_{i=1,\ldots,N}$, where $\x_i\in\R^p$ and $y_i\in\{-1,1\}$, i.e., positive or negative; yes or no.
We want to find a rule which best separates these observations into the correct classes $\{-1,1\}$, as well as being able to classify new, unseen observations $\x_{\text{new}}$ of the same form.
Some observations are hard to classify, whereas some are not.
One way to look at binary classification is to imagine the $p$-dimensional space of the observations $\x$, and think of the classifier as finding the line which best splits the observations into their corresponding label. 
Some observations are not at all close to the boundary, and so they are easily classified.
The problems start when the observations are close to the boundary.
\citet{adaboost} realized that one could assign a weight to each observation.
First, assign equal weight to each observation.
Then, use a weak learner $h(\cdot)$ to make an initial classifier, minimizing the weighted sum of misclassified points.
Finally, based on the misclassification rate of this classifier, calculate a weight $\alpha$ to give to this classifier.
Currently, the classifier is $F_1(\cdot)=\alpha_1h_1(\cdot)$.
After this initial classification, some points will be correctly classified, and some will be misclassified.
We increase the weights of the misclassified ones, and normalize the weights afterwards.
This then also results in the correctly classified ones having a reduced weight.
In the next iteration, apply again a weak learner which minimizes the weighted sum of the observations and reweight observations accordingly as before.
Again, calculate a weight to give to this new classifier, and add it to the previous classifier, such that $F_2(\cdot)=\alpha_1h_1(\cdot)+\alpha_2h_2(\cdot)$.
Continue iterating in this fashion until an iteration $m$.
The resulting final classifier, the AdaBoost classifier, becomes $\hat{F}(\cdot)=F_m(\cdot)=\sum_{i=1}^m\alpha_ih_i(\cdot)$.
It is a linear combination of the weak classifiers, and in essence a weighted majority vote of weak learners given the observations.

The AdaBoost algorithm often carries out highly accurate prediction. In practice, it is often used with stumps: Decision trees with one split. For example, \citet{bauer-kohavi} report an average 27\% relative improvement in the misclassification error for AdaBoost using stump trees, compared to the error attained with a single decision tree. They conclude that boosting not only reduces the variance in the prediction error from using different training data sets, but that it also is able to reduce the average difference between the predicted and the true class, i.e., the bias. \citet{breiman1998} supports this analysis. Because of its plug-and-play nature and the fact that it never seemed to overfit (overfitting occurs when the learned classifier degrades in test error because of being too specialized on its training set), Breiman remarked that ``boosting is the best off-the-shelf classifier in the world'' \citep{ESL}.

%Overfitting occurs when the out-of-sample error starts to increase. At this point, the model is starting to be too sensitive to the structure of the specific data set it is estimated on. One way of thinking about it is that it is starting to fit to the error terms. Since what we actually care about is the performance on a test set, we want to stop just before the model starts overfitting.
While originally developed for binary classification, boosting is now used to estimate the unknown quantities in more general statistical models and settings.
We therefore extend our discussion to a more general statistical regression scheme.
In its original formulation, the AdaBoost classifier does not have interpretable coefficients, and as such it is a so-called black-box algorithm.
This means that we are unable to infer anything about the effect of different covariates.
In statistics, however, we are interested in models which are interpretable.

%See some figure for a schematic overview of the algorithm.

\section{General model structure, setting, and chosen notation} %Statistical model fitting and setting
The aim of statistical boosting algorithms is to estimate and select the effects in structured additive regression models.
Consider a data set
\begin{equation}
    D=\{\x^{(i)},\y^{(i)}\}_{i=1,\ldots,N}
\end{equation}
containing the values of an outcome variable $\y$ and predictor variables $\x_1, \x_2,\ldots,\x_p$, forming covariate matrix $\x=(\x_1, \x_2,\ldots,\x_p)$.
We assume that the samples $i=1,\ldots,n$ are independently generated from an identical distribution over the joint space $\setX\times\setY$.
The input space of $\x$ is a possibly high-dimensional $\setX\in\R^p$ and the output space is a low-dimensional space $\setY$.
For the majority of applications, the output space $\setY$ is one-dimensional, but we will explicitly allow for multidimensional outcome variables.
Our objective is to model the relationship between $\y$ and $\x$ and to obtain an ``optimal'' prediction of $\y$ given $\x$.
To model the relationship, we will use an approach which is similar to the generalized additive model (GAM) approach \citep{gam-book}.
We will assume that the conditional outcome $\y|\x$ follows some probability distribution function (pdf)
\begin{equation}\label{eq:psi}
    \psi(\y|\theta(\x)),
\end{equation}
where $\theta$ is a parameter in the distribution function, typically related to the mean. We will at times refer to $\psi$ as a prediction function, when we use it to estimate parameters.
Further, we will model $\theta$ as a functional of the covariates $\X$, with conditional expectation given the observed value $\x$ as
\begin{equation}
    g(\mathbb{E}(\theta(\x))=f(\x),
\end{equation}
where $g(\cdot)$ is a so-called link function and $f(\cdot)$ is an additive predictor.
We see that if we use $g^{-1}(\cdot)$, the inverse of the link function, on this expression, we get
\begin{equation}
    \mathbb{E}(\theta(\X=\x))=g^{-1}(f(\x)).
\end{equation}
This means that the conditional expectation of $\theta$ given the observed $\x$ is a transformation of the additive predictor $f(\x)$ using the inverse of the link function.
The link function will be chosen appropriately for the parameter $\theta$ in the distribution $\psi$, and is typically used to constrain the domain of the parameter.
For example, if we choose the logarithm as the link function, the inverse link function is the exponential function, meaning that
\begin{align}
    \mathbb{E}(\theta(\X=\x)|\X=\x)=\exp(f(\x)),
\end{align}
which will constrain the expectation to be a positive number.
The predictor $f(\cdot)$ can be modeled in many ways.
A common model is to let it be an additive predictor, consisting the additive effects of the single predictors.
This is called a GAM, and is specified by
\begin{equation}\label{eq:gam}
    f(\x)=\beta_0+f_1(x_1)+\ldots+f_p(x_p),
\end{equation}
where $\beta_0$ is a common intercept and the functions $f_j(x_j),j=1,\ldots,p$ are single predictors, which are the partial effects of the variables $x_j$.
The generic notation $f_j(x_j)$ may be different types of predictor effects such as classical linear effects $x_j\beta_j$, smooth non-linear effects constructed via regression splines, spatial effects or random effects of the explanatory variable $x_j$, and so on.%\todo[inline]{add citations on base learners here?}
In statistical boosting algorithms, we typically use component-wise effects, meaning that the different partial effects are estimated by separate base-learners $h_1(\cdot),\ldots,h_p(\cdot)$. Read more about this in section \ref{sec:component} on component-wise boosting. The component-wise effects will typically be built up by additive estimation of base-learners, and statistical boosting is one way to perform this additive estimation.

We evaluate the fit of the model and its additive predictor using a loss function $\rho(y,f(\cdot))$, which is a measure of the discrepancy between the observed outcome $\y$ and the additive predictor $f(\cdot)$.
In machine learning and optimization, one usually talks of loss functions, and as the name reveals, we wish to minimize this.
We can use the negative log-likelihood of the distribution of the response as a loss function because it is common in statistics \citep{mayr14a}. In these cases, the loss function, which works on one set of observations $(\x_i,\y_i)$, is
\begin{equation}
    \rho(\y_i,\theta(\x_i))=-\log{\psi(\y_i|\theta(\x_i)},
\end{equation}
since the likelihood of one observation is simply the distribution given the observed data.
Note that maximizing the log-likelihood is equivalent to minimizing the Kullback-Leibler Divergence, which is a measure of the difference between the distribution of the data itself and the assumed distribution $\psi$.

\subsection{Example of a model and corresponding loss function}
Let us look at a specific example of a setting, using the notation described above.
We have a dataset $D=\{\x_i,y_i\}_{i=1}^N$ where the responses $y_i$ are continuous, and which we assume follow a normal distribution given the data.
Thus we wish to model the conditional mean $\mu$, and so we use $\mu$ in place of $\theta$.
Since the responses are continuous and normal, we do not need any transformation of the additive predictor, which means that the link function is the identity function,
\begin{equation}
    g(\x)=\x.
\end{equation}
Further, it means that
\begin{equation}
    \mathbb{E}(\mu(\x))=f(\x).
\end{equation}
For a normally distributed observation $y$, the likelihood is the pdf,
\begin{equation}
    .f(y|\mu,\sigma^2)=\frac{1}{\sqrt{2\pi\sigma^2}}\exp\left\{-\frac{(y-\mu)^2}{2\sigma^2}\right\},
\end{equation}
and we derive the loss function $\rho$ accordingly, yielding
\begin{align*}
    \rho(y,\mu(\x))&=-\log{f(y|\mu(\x))}\\
    &=\log{(\sqrt{2\pi\sigma^2})}+\frac{(y-\mu(\x))^2}{2\sigma^2} \\
    &\propto(y-\mu(\x))^2,
\end{align*}
which is the familiar $L_2$ loss function. Note that since we will only model $\mu(\cdot)$, the loss function need not depend on $\sigma^2$. With all those parts in place, we can model the additive predictor $f(\cdot)$.

\subsection{Model selection and model assessment}
\textit{Test error}, also referred to as \textit{generalization error}, is the prediction error over an independent test sample
\begin{equation}\label{eq:test-error}
    \text{Err}_{\tau}=\mathbb{E}[L(Y,\hat{f}(X))|D]
\end{equation}
where both $X$ and $Y$ are drawn randomly from their joint distribution (population).
Here the training set $D$ is fixed, and test error refers to the error for this specific training set.
A related quantity is the expected prediction error (or expected test error)
\begin{equation}
    \text{Err}=\mathbb{E}[L(Y,\hat{f}(X))]=\mathbb{E}\left[\text{Err}_D\right]
\end{equation}
Note that this expectation averages over everything that is random, including the randomness in the training set that produced $\hat{f}$.

Estimation of $\text{Err}_D$ will be our goal, although we will see that $\text{Err}$ is more amenable to statistical analysis, and most methods effectively estimate it.

\textit{Training error}, sometimes called empirical risk, is the average loss over the training sample
\begin{equation}
    R=\overline{\text{err}}=\frac{1}{N}\sum_{i=1}^NL(y_i,\hat{f}(x_i)).
\end{equation}
We would like to know the \textit{expected} test error of our estimated model $\hat{f}$,
\begin{equation}
    \mathbb{E}[R]=\mathbb{E}[\overline{\text{err}}]=\mathbb{E}\left[\frac{1}{N}\sum_{i=1}^NL(y_i,\hat{f}(x_i))\right].
\end{equation}
As the model becomes more and more complex, it uses the training data more and is able to adapt to more complicated underlying structures.
Hence there is a decrease in bias but an increase in variance.
The bias here refers to assumptions made on the structure of the response, which is not based on the data.
An example of a model with high bias is a linear model
\begin{equation}
    y=\beta_0+\sum_{j=1}^p\beta_jx_j.
\end{equation}
It has high bias because it is only able to model a response $y$ which is linear in each covariate $x_j$.
A model with high variance, on the other hand, does not make many assumptions on the relationship.
Given two observations $\x_1$ and $x_2$, which are relatively similar, a model with high variance may propose two responses which are relatively different from each other.
At least more so than a model with high bias, but low variance.
There exists some intermediate model complexity that gives minimum expected test error, which is what we want to achieve.

Unfortunately training error is not a good estimate of the test error.
Typically, the training error will drop to zero if we increase the model complexity enough.
This is because, as the bias becomes very small and the variance very large, the model ends up effectively learning the randomness inherent in the data, and not the structure of the relationship between the input $X$ and the output $Y$.
This is what is called overfitting.
An overfit model generalizes poorly, and it is generalization that we are after.

So how do we estimate the expected test error of a model $\hat{f}$?
Typically our model will have a tuning parameter $\alpha$, and so we write $\hat{f}_\alpha$.
This tuning parameter varies the complexity of our model, and we wish to find that value of $\alpha$ which minimizes error, i.e., which produces the minimum of the average test error.

It is important to note that there are two separate goals that we have in mind.
First, we wish to perform \textbf{model selection}.
We wish to choose the best model, and to do this we need to estimate the performance of different models.
This is typically done in two steps.
First, the model $\hat{f}_\alpha$ needs to be estimated.
This is done using what is called a \textit{training set}.
Second, we estimate the prediction error of this model on a \textit{validation set}.
Having done so, for various values of $\alpha$, we wish to perform \textbf{model assessment}, to find the $\alpha$ which minimizes generalization error.
We wish to, as accurately as possible, estimate the prediction error (generalization error) of our chosen model.
This is typically done using a so-called \textit{test set}.
It is important that we do not consider the performance of our chosen model on the validation set as the prediction error.
This estimate is biased, because we are only looking at the performance of this model \textit{because} its performance is good.
Ideally, the test set should be kept completely separate, and be brought out only at the end of the data analysis.

In a setting where data is abundant, one can simply split the data into these three parts: Training, validation, and test.
However, often, data is not abundant, and we wish to use the data as efficiently as possible.
There exist good methods to do this.
Notable general methods are the bootstrap, and K-fold cross-validation.
We will in this thesis not discuss the bootstrap, but we will discuss cross-validation in- depth later, in subsection \ref{subsec:K-fold}.
In short, however, K-fold cross validation uses a training data set and uses this as a joint training and validation set.
First, split the data set into a test set and an original training set.
Then, split the original training set into $K$ equally sized parts.
For each part, \textit{train/estimate} the model $\hat{f}_\alpha$ on $K-1$ of the parts, and \textit{validate} on the $K$-th (the last) part.
Then do this $K$ times, treating each part in turn as a validation set.
Then sum up the validation errors across the $K$ validation sets.
This means that the entire, original training set has also been used as a validation set.
There is still a bias, but it is smaller than validating on the same data as training. 

Our goal will be to use gradient boosting to minimize \ref{eq:test-error}.
To understand gradient boosting, we first need to understand the gradient descent algorithm.

\section{Gradient descent}
Suppose we are trying to minimize a differentiable multivariate function $G\colon\R^m\to\R$, where $m\in\N$. Gradient descent is a greedy algorithm for finding the minimum of such a function $G$, and one which is quite simple and surprisingly effective. If all partial derivatives of $G$ at a point $\x=(x_1,x_2,\ldots,x_m)$ exist, then the gradient of $G$ at $\x$ is the vector of all its partial derivatives at $\x$, namely
\begin{equation}
    \nabla G(\x)=\left(\frac{\partial G(\x)}{\partial x_1},\frac{\partial G(\x)}{\partial x_2},\ldots,\frac{\partial G(\x)}{\partial x_m}\right).
\end{equation}
The motivation behind the gradient descent algorithm is that in a small interval around a point $\x_0\in\R^m$, $G$ is decreasing the most in the direction of the negative gradient at that point. Therefore, by taking a small step slightly in the direction of the negative gradient, from $\x^{[0]}$ to a new value $\x^{[1]}$, we end up with a slightly lower function value: The new function value $G(\x^{[1]})$ will be less than $G(\x^{[0]})$. In some versions of the algorithm, the step length $\nu\in(0,1]$ is found by a line search, i.e., by finding the step length which gives the best improvement. In other versions, one simply uses a fixed step length. The gradient descent algorithm repeats this procedure until convergence. Indeed, with a sufficiently small step length, gradient descent will always converge, albeit possibly to a local minimum. For a schematic overview of the algorithm, see Algorithm \ref{algo:grad-desc}.
\begin{algorithm}
\caption{Gradient descent}
\label{algo:grad-desc}
We want to minimize $G(\x)$, i.e. solve $\min_{\x}G(\x)$.
\begin{enumerate}
    \item Start with an initial guess $\x^{[0]}$, for example $\x^{[0]}=\0$, and set $m$ to 0.
    \item\label{grad-desc-iter} Increase $m$ by 1.
    \item Calculate the direction to step in, $\g_{m-1}=-\nabla G(\x^{[m-1]})$.
    \item Solve the line search to find the best step length $a^{[m]}$,
        \begin{equation*}
            a^{[m]}=\argmin_{a}\x^{[m-1]}+a\cdot\g_{m-1}.
        \end{equation*}
    \item The step in iteration $m$ becomes $\h_m=a^{[m]}\cdot\g_{m-1}$.
    \item Let $\x^{[m]}=\x{[m-1]}+\h_{m-1}$.
    \item If $m<\mstop$, go to step \eqref{grad-desc-iter}.
    \item The resulting minimum point is $\x^{[\mstop]}=\x^{[0]}+\sum_{m=1}^M\h_m(\x^{[m]})$.
\end{enumerate}
\end{algorithm}
The gradient descent algorithm is surprisingly robust. Even though it may converge to a local minimum, it often seems to find good solutions globally. This is likely related to research which has found that in high-dimensional spaces, most minima are not minima, but in fact, saddlepoints masquerading as local minima \citep{saddlepoints}. This means that training will slow since the gradient will be small at this saddlepoint or plateau. When using a gradient descent method typically one sets a threshold at which the algorithm terminates when the gradient becomes smaller than the threshold. However if powering through the saddlepoint, then the multivariate gradient descent search should be able to continue digging downwards from these points.

\section{The gradient boosting approach}
\subsection{Gradient boosting}
In a seminal paper, \citet{friedman2001} developed an iterative algorithm for fitting an additive predictor \eqref{eq:gam}.
He called the algorithm gradient boosting.
He showed that AdaBoost performs this algorithm for a particular loss function, namely the exponential loss function.
This provided a connection between what had previously been in the machine learning domain, with the statistical domain.
See \citet{ESL} for a good demonstration of this argument.
Friedman's work provided a way of viewing boosting through a statistical lens, and connected the successful machine learning approach to the world of statistical modelling.
Gradient boosting can be viewed as an optimization procedure in functional space, because gradient boosting is gradient descent in the functional parameter space spanned by the base learners \citep{friedman2001, buhlmann2007}.

Consider the task of deriving a general prediction $\hat{\eta}$ by minimizing the expectation of a loss function $\rho(\cdot,\,\cdot)$
assumed to be differentiable with respect to $\eta$:
\begin{equation}\label{eq:min}
    \hat{\eta}=\argmin_\eta\left(\mathbb{E}_{Y,X}[\rho\{Y,\eta(X)\}]\right),
\end{equation}
where $Y$ and $X$ are the random variables for response and covariates, respectively.
$\eta$ here denotes a general prediction.
In practice, to estimate a predictor based on a sample of observations, we minimize the empirical risk $R_D$ of the data set
$D=(\x_i,y_i)_{i=1}^n$,
\begin{equation}
    R_D(\eta)=\frac{1}{n}\sum_{i=1}^n \rho(y_i,\eta(\x_i)).
\end{equation}
To avoid overfitting, we impose sufficient regularization. This is easy to do in a gradient boosting algorithm.

The key idea of boosting is to iteratively fit the different predictors with simple functions (base-learners) and combine the estimates to a predictor.
The base-learners are in particular fitted to the negative gradient of the loss function.

Now, consider the problem of finding a predictor which minimizes the empirical risk of a chosen loss function on a data set
$D=\{x_i,y_i\}_{i=1}^N$,
\begin{equation}\label{eq:argmin-eta}
    \eta^*=\argmin_{\eta}R_D(\eta)=\argmin_{\eta}\sum_{i=1}^n\rho(\y_i,\eta(\x_i)),
\end{equation}
where the parameter $\eta$ is a predictor.
The gradient boosting algorithm is a way to build up such a predictor $\eta$ by way of iterative fitting of base learners $h$.

The approach proposed by \citet{friedman2001} is to take inspiration from gradient descent and let $\hat{\eta}$ be a sum
\begin{equation}
    \hat{\eta}=\beta_0+\sum_{m=1}^{m_{\text{stop}}}f^{[m]},
\end{equation}
where the first term, $\beta_0$, is an initial guess, usually $\overline{y}$, and the remaining set $\{\hat{f}^{[m]}\}_{m=1}^M$ of 
functions is a set of increments -- steps, or boosts -- defined by the optimization method. Note that this structure
is the same as the decomposition of the solution to the gradient descent algorithm.

To perform gradient descent on our objective function, the empirical risk $R_D(\cdot)$,
we need to compute the negative gradient of the loss function, with respect to the additive predictor. For a response $y$ dependent on
covariates $\x$, this is
\begin{equation}
    -\frac{\partial}{\partial\eta} \rho(y,\eta(x)),
\end{equation}
We denote the realization of the negative gradient on the observed data by $\u$, and call $\u$ \textit{generalized residuals}, and they are
\begin{equation}
    \u=\left(u_i\right)_{i=1}^N=\left(-\frac{\partial}{\partial \eta}\rho(y_i, \hat{\eta})\big\rvert_{\hat{\eta}=\hat{\eta}^{[m-1]}(x_i)}\right)_{i=1}^N,
\end{equation}
where $\hat{\eta}^{[m-1]}$ is the estimate of the additive predictor at the previous iteration.
We now have a vector of generalized residuals, where each element $i$ is a measure of the error that the model in iteration $m-1$ makes
in trying to predict the outcome $y_i$ given $\x_i$.
With the generalized residuals in hand, we should now be able to perform a gradient descent step, which should lead us closer to a solution
to the minimization problem. We can treat each $\eta(x_i)$ as a parameter to optimize. By using the gradient descent algorithm directly, we
can then calculate the optimal step to take, and add an increment to the estimate $\hat{\eta}(x_i)$, for each $i=1,2,\ldots,n$.

\subsection{Functional Gradient Boosting}\label{sec:FGD}
However, while this nonparametric approach would reduce the error of each data point, it will not generalize to a similar data set where one observes different data points.
We are only looking at the observed data points, and not at neighboring points in $\setX$ space.
We have to keep in mind that although we are optimizing the empirical risk over a specific data set,
we are actually trying to minimize the expected value in XXXX, over all values of $\X$ and $Y$ in the joint distribution.
Additionally, we wish to have an interpretable model.
\todo[inline]{Find correct equation to reference for minimization}
Therefore, we must impose smoothness to neighboring points in the $\setX$ space. We can do this by choosing steps of
(parameterized) \textit{functions} instead of steps of function \textit{values}.
Therefore, since the solutions are parameterized functions, and we are performing gradient descent,
the approach by \citet{friedman2001} develops a \textit{functional} gradient descent (FGD) algorithm.
Each parameterized step is a realization of a so-called base learner
\begin{equation}
    h(\cdot).
\end{equation}
A base learner is usually a relatively simple parametric effect of $\beta$.
Again, typical examples are linear least squares, stumps (trees with one split; see \citet{buhlmann2007} and \citet{ESL}), and splines with a few degrees of freedom.
There are several reasons to use simple base learners.
One is that there often exists fast methods for estimating a single base learner.
Therefore there will be little computational cost in each step. 
econdly, there is relatively more to gain by combining simple learners,
rather than combining complex learners.

For the functional gradient descent, we still start with an initial value for $\eta$, a constant $\beta_0$.
Then we iterate, let us say, at each step $m>0$ first calculating the generalized residuals of the previous iteration,
\begin{equation}
    \u^{[m-1]}=\left(-\frac{\partial}{\partial \eta}\rho(y_i, \eta)\big\rvert_{\hat{\eta}=\hat{\eta}^{[m-1]}(x_i)}\right)_{i=1}^N,
\end{equation}
like we have seen before. Here we insert the model from the previous step, $\hat{\eta}^{[m-1]}$.
We now perform a gradient descent step, but we are constrained to steps which are functions of the base learner $h(\cdot)$.
In a functional sense, to take the steepest gradient, we must choose the relization of the base learner which produces the function $\hat{h}^{[m]}$ \textit{most parallel} to $\u^{[m-1]}$.
Another way of looking at it is that it this is the $h$ most  correlated with $\u^{[m-1]}$ over the data distribution.
This means that this $\hat{h}^{[m]}$ is an approximation of the generalized residuals $\u^{[m-1]}$, or, a projection of the generalized residuals onto the space spanned by the base learner function class.
We obtain that $\hat{h}_m$ by fitting the base learner $h(\cdot)$ to the generalized residuals.
The method of fitting depends on the base learner.
If, for example, the base learner is a linear regressor, then this will be $\hat{h}^{[m]}=\left(\u^{[m-1]}\right)^T\hat{\bbeta}^{[m]}$, where
\begin{equation}
    \hat{\bbeta}^{[m]}=\left(\left(\u^{[m-1]}\right)^T\u^{[m-1]}\right)^{-1}\left(\u^{[m-1]}\right)^T\y.
\end{equation}
Having estimated the base learner, we do a line search to find the appropriate step length to use in order to minimize the loss function the most,
\begin{equation}
    a^{[m]}=\argmin_{a}R_D\left(\hat{f}^{[m-1]}+a^{[m]}\cdot\hat{h}^{[m]}{\cdot}\right).
\end{equation}
We add the estimated learner times the step length to the current model, obtaining
\begin{equation}
    \hat{f}^{[m]}(\cdot)\gets \hat{f}^{[m-1]}(\cdot)+a^{[m]}\hat{h}^{[m]}(\cdot).
\end{equation}
We iterate this procedure until some stopping criterion. The resulting model
\begin{equation}
    \hat{f}_{\text{FGD}}(\cdot)=\hat{f}^{[\mstop]}(\cdot)
\end{equation}
has an additive structure which is a direct effect of the gradient descent algorithm, as the aggregation of base learners is strictly additive:
In every iteration, small increments are added to the additive predictor.
For a schematic overview, see Algorithm \ref{algo:fgd}.
\begin{algorithm}
\caption{Gradient boosting, or, generic Functional Gradient Descent (FGD)}
\label{algo:fgd}
\begin{enumerate}
    \item Start with a data set $D=\{x_i, y_i\}_{i=1}^N$ and a chosen loss function $\rho(y,\hat{f}(x))$, for which we wish to
        minimize the empirical risk, i.e., the loss function evaluated on the samples,
        \begin{equation}
            \hat{f}=\argmin_{f}R(f)=\argmin_{f}\sum_{i=1}^n\rho(y_{i},f(x_{i}).
        \end{equation}
    \item Set iteration counter $m$ to 0. Initialize the additive predictor by setting $\hat{f}_0(\cdot)$ to a constant $\beta_0$. One option is to find the the best constant my numerical maximization, i.e.,
        \begin{equation}
            \beta_0(\cdot)=\argmin_c R(c).
        \end{equation}
    \item Specify a base learner class $h$, e.g. linear least squares.
    \item Increase $m$ by 1.
    \item Compute the generalized residuals (the negative gradient vector) of the previous iteration,
        \begin{equation}
            \u^{[m-1]}=\left(-\frac{\partial}{\partial f}\rho(y_i, f(x_i))\big\rvert_{f=\hat{f}^{[m-1]}}\right)_{i=1}^N
        \end{equation}
    \item Fit base learner $h$ to the generalized residuals $\u$ to obtain a fitted version $\hat{h}^{[m]}$.
    \item Find best step length for $a^{[m]}$ by a line search:
        \begin{equation*}
            a^{[m]}=\argmin_{a}R\left(\hat{f}^{[m-1]}+a\cdot\hat{h}^{[m]}{\cdot}\right).
        \end{equation*}
    \item Update $\hat{f}^{[m]}(\cdot)\gets \hat{f}^{[m-1]}(\cdot)+a^{[m]}\cdot \hat{h}^{[m]}(\cdot)$.
    \item Repeat steps 4 to 8 (inclusive) until $m=\mstop$.
    \item Return $\hat{f}(\cdot)=\hat{f}^{[\mstop]}(\cdot)=\sum_{m=0}^{\mstop}f^{[m]}(\cdot)$.
\end{enumerate}
\end{algorithm}

\subsection{Tuning parameters}
\subsubsection{Step length}
In the original generic functional gradient descent algorithm, the step length $a_m$ for each iteration is found by a line search.
\citet{friedman2001} says that fitting the data too closely may be counterproductive, and result in overfitting.
To combat the overfitting, one constrains the fitting procedure.
This constraint is called regularization.
Friedman therefore, later in the paper, proposes to regularize each step in the algorithm by a common learning rate, $0<\nu\leq1$.
Another natural way to regularize would have been to control the number of terms in the expansion, i.e., number of iterations, $M$.
However, it has often been found that regularization through shrinkage provides superior results \citep{copas1983}.

As we will see, most modern boosting algorithms omit the step of the line search to find $a_m$, but instead uses a fixed learning rate/step length $\nu$.  
The choice of this step length is not of critical importance as long as it is sufficiently small \citep{schmid-hothorn}, i.e., produces sufficient shrinkage, but the convention is to use $\nu=0.1$ \citep{mayr14a}.
This reduces the complexity of the algorithm, and reduces the number of tuning parameters to 1, the number of steps, or iterations.
There will of course be a tradeoff between the number of iterations $M$ and the size of the step length $\nu$, which is another reason to use the conventional step length each time.

\subsection{Number of iterations}\label{subsec:iterations}
With a fixed step length (learning rate), the main tuning parameter for gradient boosting is the number of iterations $M$ that are performed before the algorithm is stopped.
Its value is critical:
If $M$ is too small, the model will underfit and it cannot fully incorporate the influence of the effects on the response and will consequently have poor performance.
On the other hand, too many iterations will result in overfitting, leading to poor generalization.

%\subsection{Practical considerations}
%When boosting, one must (or should) center and scale the matrix $X$.

%\section{likelihood-based boosting}
%Lorem ipsum. \citep{DeBin2016} \citep{gamboost}.

\section{$L_2$Boost}
With the generic functional gradient boosting algorithm seen in section \ref{sec:FGD}, it is quite straightforward to derive specific algorithms to use for specific models:
It is just a matter of plugging in a chosen loss function and deriving its negative gradient.
This gives great flexibility.

In the original paper \citep{friedman2001}, Friedman derived an algorithm for the standard regression setting, which he called $L_2$Boost.
$L_2$Boost is a computationally simple variant of boosting, constructed from a functional gradient descent algorithm for the $L_2$ loss function,
\begin{equation*}
    \rho(y, \hat{y})=\frac{1}{2}(y-\hat{y})^2.
\end{equation*}
The reason it is simple is that the generalized residual $u_i$ of an observation $y_i,x_i$, i.e., the negative derivative of the loss function with regard to an estimate $\hat{y}_i=\hat{f}(x_i)$, is
\begin{equation*}
    -\frac{\partial}{\partial\hat{y}}\rho(y_i, \hat{y}_i)=y_i-\hat{y}_i,
\end{equation*}
that is, the so-called residual. The negative gradient vector $\u$ then becomes simply the residual vector,
\begin{equation*}
    \u=\left(\frac{\partial\loss(y,f(\x))}{\partial x_i}\right)_{i=1}^n=(y-f(x_i))_{i=1}^n,
\end{equation*}
and hence the algorithm results in repeated refitting of residuals \citep{friedman2001,buhlmann-yu}.
With $M=2$ iterations, this had in fact been proposed already, under the name of ``twicing'' \citep{tukey}.
See Algorithm \ref{algo:L2} for an overview of the algorithm. Note that we here use the algorithm given in \citet{buhlmann-yu}, who do not use a step length, i.e., they let $\nu_m=\nu=0.1$ for all iterations $m=1,\ldots,M$.
\begin{algorithm}
\caption{$L_2$Boost}
\label{algo:L2}
\begin{enumerate}
    \item Start with a data set $D=\{x_i, y_i\}_{i=1}^N$. Set the loss function $\rho(y,\hat{f}(x))=\frac{1}{2}(y-\hat{f}(x))^2$, for which we wish to
        minimize the empirical risk, i.e., the loss function evaluated on the samples,
        \begin{equation}
            \hat{f}=\argmin_{f}R(f)=\argmin_{f}\sum_{i=1}^n y_i-\hat{f}(x_i).
        \end{equation}
    \item Set $m=0$. Initialize $f_0(\x)$, e.g., by setting it to zero for all components, or by finding the best constant, i.e.,
        \begin{equation}
            f_0(\cdot)=\argmin_c R(c).
        \end{equation}
    \item Specify the base learner class $h$, e.g. least squares.
    \item Increase $m$ by 1.
    \item Compute the negative gradient vector, i.e., the residuals, with the model evaluated at the previous estimate
        \begin{equation}
            \u^{[m-1]}=\left(y_i-\hat{f}(x_i)\right)_{i=1}^N
        \end{equation}
    \item Estimate $\hat{h}_m$ by fitting $(\x_i,u_i^{[m-1]})$ using the base learner $h$ (like in the previous algorithm):
        \begin{equation*}
            \bbeta_m=\argmin_{\bbeta}\sum_{i=1}^N\loss(u_i^{[m-1]},h(\x_i;\bbeta))
        \end{equation*}
        This estimation can be viewed as an approximation of the negative gradient vector, and as the projection of the negative gradient vector onto the space spanned by the base learner.
    \item Update $f_m(\cdot)=f_{m-1}(\cdot)+h(\cdot;\bbeta_m)$.
    \item Repeat steps 4 to 7 (inclusive) until $m=M$.
    \item Return $\hat{f}(\cdot)=\hat{f}_{\mstop}(\cdot)=\sum_{m=0}^{\mstop}f_m(\cdot)$.
\end{enumerate}
\end{algorithm}
%They also prove some nice important theoretical results for L2Boost.
%\todo[inline]{more on L2Boost!!}
%\subsection{$L_2$Boost example}
%Lorem ipsum.

\section{High dimensions and component-wise gradient boosting}\label{sec:component}
% add this to component-wise
In modern biomedical statistics, it is crucial to be able to handle high-dimensional data. In some situations, a data set consists of more predictors $p$ than observations $N$. When $p$ is much larger than $N$ ($p>>N$), we talk about high-dimensional settings. In order to address the issue of analyzing high-dimensional data sets, a variety of regression techniques have been developed over the past years. Many of these techniques are characterized by a built-in mechanism for ``regularization'', which means that shrinkage of coefficient estimates or selection of relevant predictors is carried out simultaneously with the estimation of the model parameters. Both shrinkage and variable selection will typically improve prediction accuracy: In case of shrinkage, coefficient estimates tend to have a slightly increased bias but a decreased variance, while in case of variable selection, overfitting the data is avoided by selecting only the most informative predictors. For instance in the $L_2$Boost algorithm, if one uses a least squares base learner which uses all $p$ dimensions, we see that it is infeasible: The matrix which must be inverted is singular when the number of predictors $p$ is larger than the number of observations $N$. For other models, it might be possible to estimate parameters for each predictor, but it would very easily result in overfitting. If, for instance, the data set input $\X$ consists of gene expressions, it is obvious that the response variable $y$ is not dependent on every single gene.
%Note that regularization is not only useful in the high-dimensional data setting, but also tends to improve prediction accuracy in low-dimensional settings where $p\leq N$, where $N$ is the number of observations in the data set, and $p$ the number of predictors, i.e., columns in $\X$. With such data sets, it will be infeasible to use a base learner $h$ which incorporates all $p$ dimensions of the observation matrix $\X$. 

\subsection{Stagewise, not stepwise}
Component-wise gradient boosting is an algorithm which works very well in these settings.
In fact, Buhlmann believes that it is mainly in the case of high-dimensional predictors that boosting has a substantial advantage over classical approaches \citep{buhlmann2006}.
The component-wise approach was first proposed in the $L_2$Boost paper \citep{buhlmann-yu}, and is very much an active field of research \citep{buhlmann2006, mayr14a, mayr14b, mayr17}.
In the gradient boosting algorithm described in algorithm \eqref{algo:fgd}, we start out with an additive predictor $f^{[0]}(\cdot)$ which only consists of a constant.
We have not added any effects of covariates yet.
Instead of adding a small effect from all predictors, the component-wise approach adds only the effect of one variable at a time.
This is similar to the typical statistical model selection regime of forward stepwise model selection.
In forward stepwise, we will iterate in the following manner.
We start with an empty set of predictors, or covariates, and look at each separately. Looking at each separately means adding it to the set of predictors and estimating a model with those predictors.
Then we add, to the set of predictors, that predictor which gives the best improvement to the objective function.
We repeat this in each step, but we estimate the entire model in each step.
The main idea of component-wise gradient boosting is to do this, except in a stagewise manner.
This means that we do not change the added parameters, but we only estimate the next one.

Now, consider the problem of finding a predictor on the GAM form \eqref{eq:gam}, which minimizes the empirical risk of a chosen loss function on a data set $D=\{x_i,y_i\}_{i=1}^N$,
\begin{equation}\label{eq:argmin-eta}
    \eta^*=\argmin_{\eta}R_D(\eta)=\argmin_{\eta}\sum_{i=1}^n\rho(\y_i,\eta(\x_i)),
\end{equation}
where the parameter $\eta$ is an additive predictor
\begin{equation}
    \eta(\x)=g(\theta)=\beta_0+f(\x)=\beta_0+\sum_{j=1}^p f_j(x_j),
\end{equation}
and $g(\cdot)$ is a known, monotonic link function appropriately chosen.
Since we have component-wise effects, the model is interpretable and yields itself nicely to statistical inference, and, to a component-wise expansion of the gradient boosting algorithm.


The structure of the component-wise boosting algorithm is very much the same as the generic functional gradient boosting algorithm (algorithm \ref{algo:fgd}), but with some additional steps.
Instead of using one base learner which incorporates all predictors,
we use a set $\mathcal{H}$ of base learners consisting of a separate base learner for each component of the covariates.
These base learners may have the same structure, or they may not, but importantly, they only use a single covariate.
For example, if we use a linear least squares model as base 
learners, the set of base learners would be
\begin{equation}
    \mathcal{H}=\{h_1(\x;\beta_1)=\beta_1 x_1,h_2(\x;\beta_2)=\beta_2 x_2,\ldots,h_p(\x;\beta_p)=\beta_p x_p \}.
\end{equation}
The initialization of the algorithm is the same as in the FGD algorithm:
We first initialize the additive predictor to a constant $\beta_0$.
In a given iteration $m$, we first derive the generalized residuals by calculating 
the negative gradient where we insert the additive predictor from the previous step,
\begin{equation}
    \u^{[m-1]}=\left(-\frac{\partial}{\partial \eta}\rho(y_i, \eta)\big\rvert_{\eta=\hat{\eta}^{[m-1]}(x_i)}\right)_{i=1}^N.
\end{equation}
Note that this calculation is exactly like in the generic gradient boosting algorithm.
While the generic FGD algorithm here only estimated a single base learner, in the component-wise we now estimate all base learners separately, obtaining $p$ estimated functions
\begin{equation}
    \hat{h}_1^{[m]}(\cdot),\,\hat{h}_2^{[m]}(\cdot),\,\ldots,\,\hat{h}_p^{[m]}(\cdot),
\end{equation}
These estimated functions can again be viewed as approximations of the negative gradient vector, and as the projection of the negative gradient vector onto the space spanned by the component-wise base learner. However, this time, they are projections onto
only one component of the covariate space. To select the best-fitting base-learner $h_{j^{[m]}}^{[m]}$, we select the one with
the smallest residual sum of squares error
\begin{equation}
    j^{[m]}=\argmin_{j\in\{1,2,\ldots,p\}}\sum_{i=1}^N \left(u_i-\hat{h}_j^{[m]}\right)^2.
\end{equation}
Note that this makes sense from a linear algebra perspective: Choosing the one with minimal RSS means that we choose the one with the
smallest projection error, or the one with the most signal.
We add this best-fitting base-learner to the current model, with a pre-specified step length of $\nu$. Hence the model after iteration $m$ is
\begin{equation}
    \hat{f}^{[m]}(\cdot)\gets \hat{f}^{[m-1]}(\cdot)+\nu\cdot\hat{h}_{j^{[m]}}.
\end{equation}
In a component-wise perspective, we update the predictor of the selected component,
\begin{equation}
    \hat{f}_{j^{[m]}}^{[m]}(\cdot)\gets \hat{f}_{j^{[m]}}^{[m-1]}(\cdot)+\nu\cdot\hat{h}_{j^{[m]}},
\end{equation}
and for all other components $j\in\{j\colon j\neq j^{[m]},\,j=1,2,\ldots,p\}$, the update in iteration $m$ is simply
to keep the predictor from the last iteration
\begin{equation}
    \hat{f}_{j}^{[m]}(\cdot)\gets \hat{f}_{j}^{[m-1]}.
\end{equation}
We continue iterating until the iteration number $m$ reaches the tuned stopping iteration $m_{\text{stop}}$.
The final additive predictor becomes
\begin{equation}
    \hat{\eta}=\hat{\eta}^{[m_{\text{stop}}]}=\beta_0 + \sum_{m=1}^{m_{\text{stop}}}\nu\cdot\hat{h}_{j^{[m]}}^{[m]}(\cdot).
\end{equation}
Note that any base-learner $h_j$ can be selected at multiple iterations. The partial effect of the variable $x_j$ is the sum of the estimated corresponding base learner in all iterations where it was selected, i.e.,
\begin{equation*}
    \hat{f}_j(x_j)=\sum_{m=1}^{m_{\text{stop}}}\nu\cdot\hat{h}_j^{[m]}(x_j)\indicator\left(j^{[m]}=j\right),
\end{equation*}
where $I(\cdot)$ is an indicator function. Hence the resulting additive predictor is a sum of component-wise predictors in the GAM form of
\begin{equation}
    \hat{\eta}(\x)=\beta_0+\sum_{j=1}^p \hat{f}_j(x_j).
\end{equation}
For a schematic overview of the algorithm, see Algorithm \ref{algo:component-wise}.
\begin{algorithm}
\caption{Component-wise gradient boosting}\label{algo:component-wise}
\begin{enumerate}
    \item Start with a data set $D=\{x_i, y_i\}_{i=1}^N$ and a chosen loss function $\rho(y,f(x))$, for which we wish to
        minimize the empirical risk, i.e., the loss function evaluated on the samples,
        \begin{equation}
            \hat{f}=\argmin_{f}R(f).
        \end{equation}
    \item Set iteration counter $m$ to 0. Specify a step length $\nu$. Initialize the additive predictor to an offset by setting $f_0(\cdot)$ to a constant $\beta_0$. One option is to find the the best constant by numerical maximization, i.e.,
        \begin{equation}
            \beta_0=\argmin_c R(c).
        \end{equation}
    \item Specify a set of base learners $\mathcal{H}=\{h_1(\cdot),\dotsc,h_p(\cdot)\}$, where each $h_j$ is univariate and takes column $j$ of $\X$.
    \item\label{first-step} Increase $m$ by 1.
    \item Compute the negative gradient vector, i.e., the generalized residuals after the previous iteration of the boosted model,
        \begin{equation}
            \u^{[m-1]}=\left(-\frac{\partial}{\partial f}\rho(y_i, f(x_i))\big\rvert_{f=\hat{f}^{[m-1]}}\right)_{i=1}^N.
        \end{equation}
    \item For each base learner $h_j\in\mathcal{H},j=1,\ldots,p$, estimate $\hat{h}_{j}^{[m]}$ by fitting $(\X_i,u_i)$ using the base learner $h_j(\cdot)$. We obtain
        \begin{equation}
            \hat{h}_1^{[m]}(\cdot),\hat{h}_2^{[m]}(\cdot),\ldots,\hat{h}_p^{[m]}(\cdot).
        \end{equation}
    \item Select the best-fitting component $j^{[m]}$, i.e., with lowest RSS,
        \begin{equation}
            j^{[m]}=\argmin_{j\in\{1,2,\ldots,p\}}\sum_{i=1}^N \left(u_i-\hat{h}_j^{[m]}\right)^2.
        \end{equation}
    \item\label{last-step} Update the current model with the best-fitting model from the current iteration
        \begin{equation}
            \hat{f}^{[m]}(\cdot)\gets \hat{f}^{[m-1]}(\cdot)+\nu\cdot \hat{h}_{j^{[m]}}^{[m]}(\cdot).
        \end{equation}
    \item Repeat steps \ref{first-step} to \ref{last-step} (inclusive) until $m=m_{\text{stop}}$.
    \item Return the final boosted additive predictor
        \begin{equation}
            \hat{f}(\cdot)=\hat{f}^{[m_{\text{stop}}]}(\cdot)=\beta_0+\sum_{m=1}^{m_{\text{stop}}}\nu\cdot\hat{h}_{j^{[m]}}^{[m]}(\cdot)
        \end{equation}
\end{enumerate}
\end{algorithm}

\section{Boosting performs data-driven variable selection}\label{sec:variable-selection}
Stopping the algorithm before every base-learner was at least selected once effectively excludes all non-selected base-learners
(and thus also the corresponding covariates) from the final model.
The algorithm is therefore able to perform variable selection and model fitting simultaneously.
Furthermore, early stopping shrinks effect estimates toward zero \citep{buhlmann2007, DeBin2016}, similar to $L_1$-penalized regression
such as the lasso \citep{lasso, efron2004}.
Shrinkage of effect estimates lead to a lower variance and therefore to more stable and accurate predictions \citep{efron1975, copas1983, ESL}.


In other words, if the number of iterations $\mstop$ is small enough, the component-wise gradient boosting algorithm will carry out automatic variable selection.
In particular the base learners applied to irrelevant variables will never be considered in the updating step, and therefore many of the columns of $\x$ will not be a part of the final model.
Some predictors will have more explanatory power, or signal, than others, and so they will be selected more often.
This is because some predictors are more correlated with the output than others.

%\citep{mayr-hofner}.

\section{Selecting $\mstop$}
As we mentioned in subsection \ref{subsec:iterations}, the crucial tuning parameter in boosting is the number of iterations, $\mstop$. 
Stopping early enough performs variable selection and shrinks the parameter estimates toward zero.
In the case of $p<N$, with $m\to\infty$, the parameters in boosting will converge towards the maximum likelihood estimates \citep{DeBin2016}, i.e., minimizing the in-sample error.
We are, on the other hand, after all interested in minimizing out-of-sample prediction error (PE).
The prediction error for a given data set is a function of the boosting iteration $m$.
What we want is therefore a good method for approximating $\PE(m)$.
This can be done in a number of ways.
Many authors state that the algorithm should be stopped early, but do not go further into the details here.
Common model selection criteria such as the Akaike Information Criteria (AIC) may be used, however the AIC is dependant on estimates of the model's degrees of freedom. 
Initial versions of boosting used it, but practice showed that AIC-based stopping rules lead to overfitting \citep{mayr-hofner}.
This is problematic for several reasons.
For $\text{L}_2\text{Boost}$, \citet{buhlmann2007} suggest that $\df(m)=\trace(B_m)$ is a good approximation.
Here $B_m$ is the hat matrix resulting from the boosting algorithm.
This was, however, shown by \citet{hastie2007} to always underestimate the actual degrees of freedom.
\citet{mayr-hofner} propose a sequential stopping rule using subsampling.
However this is computationally very expensive and not really used in practice.
Instead, cross-validation, a very common method for selection of tuning parameters in statistics, is what is used in almost all cases, both in practice and in research \citep{mayr14a,mayr14b,mayr17}.
Cross-validation is flexible and easy to implement. It is somewhat computationally demanding, because it requires several full runs of the boosting algorithm.

%\subsection{Other selection methods}
%The number of iterations in the boosting procedure, $M$, is a tuning parameter. It acts as a regularizer. AIC, etc.

\subsection{K-fold cross-validation}\label{subsec:K-fold}
K-fold cross-validation \citep{lachenbruch}, or simply cross-validation, is a general method commonly used for selection of penalty or tuning parameters because it approximates the prediction error.
In cross-validation, the data are randomly split into K rougly equally sized folds.
For a given fold $k$, all folds except $k$ act as the training data in estimating the model.
We often say that the $k$-th fold is left out.
The resulting model is then evaluated on the unseen data, namely the observations belonging to fold $k$.
This procedure is repeated for all $k=1,\ldots,K$.
An estimate for the prediction error is obtained by averaging over the test errors evaluated in each left-out fold. Let $\kappa(k)$ be the set of indices for fold $k$.
The cross-validated estimate for a given $m$ then becomes
\begin{equation}
    \CV(m)=\frac{1}{K}\sum_{k=1}^K\sum_{i\in\kappa(k)}\rho(y_i,\hat{y}_i^{-\kappa(k)}).
\end{equation}
For each $m$, we calculate the estimate of the cross-validated prediction error $\CV(m)$.
We choose $\mstop$ to be the minimizer of this error,
\begin{equation}
    \mstop=\argmin_{m}\CV(m).
\end{equation}
Typical values for $K$ are 5 or 10, but in theory one can choose any number. The extreme case is $K=N$, called leave-one-out cross-validation, where all but one observation is used for training and the model is evaluated on the observation that was left out. In this case, the outcome is deterministic, since there is no randomness when dividing into folds.

\subsection{Stratified cross-validation}
When dividing an already small number of survival data observations into $K$ folds, we might risk getting folds without any observed deaths, or in any case, very few. In stratified cross validation, we do not divide the folds entirely at random, but rather, try to divide the data such that there is an equal amount of uncensored data in each fold.
As before, let $\kappa(k)$ be the set of indices for fold $k$.
Divide the observed data into $K$ folds, as with usual cross validation, to get an index set $\kappa_{\delta=1}(k)$ for a given $k$. 
Similarly, divide the censored data into $K$ folds, obtaining $\kappa_{\delta=0}(k)$.
Finally, $\kappa(k)$ is the union of these sets: $\kappa(k)=\kappa_{\delta=1}(k)\cup\kappa_{\delta=0}(k)$.
For a detailed description of 10-fold cross-validation issues in the presence of censored data, see \citet{kohavi}.

\subsection{Repeated cross-validation}
The randomness inherent in the cross-validation splits has an effect on the resulting $\mstop$.
This is true for boosting in general, but it is especially true for real-life survival data.
In typical survival time data sets one typically has a small effective sample size (number of observed events).
We can easily imagine that we can end up with quite different values for $\mstop$ for two different splits of the data.
It has been very effectively demonstrated that the split of the folds has a large impact on the choice of $\mstop$ \citep{seibold}.
To reduce the impact of this issue, \citet{seibold} suggest to simply repeat the cross-validation scheme a few times and average the results.
They show that repeating the cross-validation procedure even only 5 times effectively averages out the randomness.
In other words, we divide the data into $K$ folds, and repeat this $J$ times.
Now let $\kappa(j, k)$ be the $k$-th fold in the $j$-th split.
We end up with a new estimate for the prediction error,
\begin{equation}
    \RCV(m)=\frac{1}{J}\sum_{j=1}^J\frac{1}{K}\sum_{k=1}^K\sum_{i\in\kappa(j,k)}\rho(y_i,\hat{y}_i^{-\kappa(j,k)}).
\end{equation}
As before, we choose $\mstop$ to be the minimizer of this error,
\begin{equation}
    \mstop=\argmin_{m}\RCV(m).
\end{equation}
In practice, to ensure we find the minimizing $m$, we let the boosting algorithm run for $m=1$ to $m=M$, where $M$ is a large number that we are sure will result in a overfitted model.

\section{Multidimensional boosting: Cyclical component-wise}
A limitation of all the classical boosting methods we have described earlier, as well as of $L_1$-penalized estimation such as the lasso method \citep{lasso}, is that they are designed for statistical problems involving a one-dimensional prediction function.
By only considering such functions, we are restricted to estimating models which only model a single quantity of interest, which is almost always the mean.
In many applications, modelling only one parameter will not be sufficient \citep{kneib2013}.
We want to be able to estimate more general models, in which more quantities, e.g. the drift and the threshold of the models described in section \ref{sec:FHT}, can be explained by covariates.
Typical examples of multidimensional estimation problems are classification with multiple outcome categories and regression models for count data.
Another example is estimating models in the GAMLSS family \citep{gamlss}.
GAMLSS, which refer to ``generalized additive models for location, scale and shape,'' are a family of models that relates not only the mean, but all parameters of the outcome distribution to the available covariates.
GAMLSS are in this sense an extension of GAM models \citep{gam-book}.
A gradient boosting algorithm called \textit{gamboostLSS} was developed for boosting such models \citep{gamboostlss-paper}.
The algorithm framework used in \textit{gamboostLSS} is inspired by the multidimensional boosting algorithm first introduced in \citet{schmid}.
We will here explain the \textit{gamboostLSS} algorithm, as presented in \citet{gamboostlss-paper}.

\section{GAMLSSBoost Algorithm}\label{sec:gamlssboost}
A key feature of GAMLSS is that every parameter of the conditional response distribution is modelled by its own predictor and associated link function.
Traditional GAMs \citep{gam-book} are typically restricted to modelling the conditional mean of the response variable, and treats possible other distributional parameters as fixed.
GAMLSS, on the other hamd, allows for regression of each distribution parameter on the covariates.
Common distribution parameters are location, scale, skewness and kurtosis, but degrees of freedom (of a $t$-distribution) and zero inflation probabilities can be modelled as well \citep{gamboostlss-paper}.
Thus, in the GAMLSS approach, the full conditional distribution of a multiparameter model is related to a set of predictor variables of interest.
Similarly to in GAMs, in GAMLSS the structure of each predictor is assumed to be additive.
Hence a wide variety of functional predictors can be included in each predictor.
Examples include non-parametric terms based on penalized splines, varying-coefficient terms and spatial and subject-specific terms for repeated measurements.
The estimation of GAMLSS coefficients is usually based on penalized likelihood maximization; for details on fitting procedures, see \citet{gamlss}.

The GAMLSS model class assumes observations $y_i$ for $i=1,2,\ldots,n$ that are conditonally independent given a set of covariates and after having accounted for spatiotemporal effects.
The conditional density
\begin{equation}\label{gamlss-density}
    f_{\text{dens}}(y_i|\btheta_i),
\end{equation}
may depend on $K$ distribution parameters
\begin{equation}
    \btheta_i=\left(\theta_{i,1},\theta_{i,2},\ldots,\theta_{i,K}\right)^T.
\end{equation}
Each distribution parameter $\theta_k,k=1,2,\ldots,K$ is modelled by its own additive predictor $\eta_{\theta_k}$ and depends additively
on the covariates.
$\theta_k$ is linked to a predictor by a known monotonic link function
\begin{equation}
    g_k(\cdot).
\end{equation}
Letting $p_k$ be the number of covariates to be used for distribution parameter $\theta_k$,
\begin{equation}
    x_{k,1},x_{k,2},\ldots,x_{k,p_k}
\end{equation}
are the covariates in the model of the parameter $\theta_k$.
A GAMLSS is given by the equations
\begin{equation}
    \eta_k\coloneqq g_k(\theta_k)=\beta_{k,0}+\sum_{j=1}^{p_k}f_{k,j}(x_{k,j}),
\end{equation}
for all $k=1,2,\ldots,K$. Here $\beta_{k,0}$ is the intercept for distribution parameter $\theta_k$, and $f_{k,j}$ represents the type of effect that covariate $j$ has on the distribution parameter $\theta_k$, through the link function.
In the case of a simple linear regression learner, a component-wise effect of component $j$ on distribution parameter $\theta_k$ would be
\begin{equation}
    f_{k,j}(x_{k,j})=x_{k,j}\beta_{k,j},
\end{equation}
where $\beta_{k,j}$ is a parameter to be estimated.
Finally, $\eta_{k}$ is the additive predictor for $\theta_k$.
Note that a GAMLSS reduces to a GAM \citep{gam-book} in the case where the distribution parameter vector is a scalar
\begin{equation}
    \btheta_i=\theta_i=\mu_i,
\end{equation}
i.e., the conditional mean of observation $i$.

For parametric models, the unknown quantities of a GAMLSS can be estimated by maximizing the log-likelihood of an observed
data set of $n$ observations of the conditional density \eqref{gamlss-density}.
The log-likelihood contribution for one sample is
\begin{equation}
    \log\{f_{\text{dens}}(\btheta|y)\},
\end{equation}
where $y$ is the response of the sample and $\btheta$ is a vector of the distribution parameters, which will be a functional which works on the covariate vector $\x$.
Hence the total log-likelihood is
\begin{equation}
    l(\btheta)=\sum_{i=1}^N\log\left(f_{\text{dens}}(\btheta|y_i)\right),
\end{equation}
Denoting estimates of the prediction functions as $\hat{\eta}_k$, estimates of the distribution parameters $\btheta$ are then obtained from transforming back via the inverse link functions,
\begin{equation}
    \hat{\theta}_k=g_k^{-1}(\hat{\eta}_{\theta_k}),
\end{equation}
for all $k=1,2,\ldots,K$.
After the original GAMLSS paper \citep{gamlss}, a penalized likelihood approach based on modified versions of the backfitting algorithm for GAM estimation was developed by the same authors \citep{gamlssR}.
Later, however, a gradient boosting algorithm, called \textit{gamboostLSS}, was developed \citep{gamboostlss-paper}.
The algorithm \textit{gamboostLSS} uses a strategy for multidimensional boosting proposed by \citet{schmid}.
\citet{thomas2018} later coined the term ``cyclical'' to describe it and here we use this notation.
The main idea of the cyclical multidimensional boosting algorithm is to have a boosting step for each parameter $k$, in each iteration, and to successively update the predictors in each iteration, using the estimates of the other distribution parameters as offset values.
We cycle through all parameter dimensions in each boosting iteration.

In any iteration, the algorithm cycles through the different parameter dimensions $k$.
In every dimension $k$, we carry out one boosting iteration.
This boosting iteration can in principle be the same as in the generic FGD algorithm \eqref{algo:fgd}, i.e., to estimate a full base learner which incorporates all covarietes.
The \textit{gamboostLSS} algorithm, however, uses the component-wise base learner strategy, introduced in section \ref{sec:component}.
This allows for, as we saw, model fitting in high-dimensional contexts, and in general, to be often powerful.
\todo[inline]{What does Riccardo mean here? Check. Also add citation.}
To use the gradient boosting approach for a multidimensional prediction function, we need to have existing partial derivatives of the loss function with regard to each predictor.
Since we are doing a gradient descent step, we use the \textit{negative} derivative of the estimate.
Since we have a prediction function which uses a vector of distribution parameters, we must take the partial derivatives.
These negative partial derivatives are
\begin{equation}
    -\frac{\partial}{\partial\eta_k}\rho(y,\boldeta)=\frac{\partial}{\partial\eta_k}\log(f_{\text{dens}}(y|\btheta)),
\end{equation}
for all $k=1,2,\ldots,K$. As in previous algorithms, we use these negative derivatives to construct generalized residual vectors.
In this multidimensional approach, we now have $K$ partial derivatives, and so we construct $K$ different generalized residual vectors $\u_k$, $k=1,2\ldots,K$.
In a given iteration $m>0$ and for a distribution parameter $\theta_k$, we construct a generalized residual by computing the negative derivative with regard to each additive predictor $\eta_k$, and inserting the current estimate $\hat{\boldeta}^{[m-1]}$, evaluated at each observation $(x_i,y_i)_{i=1}^n$.
This yields
\begin{align*}
    \u_k^{[m-1]}&=(u_{k,1}^{[m-1]},u_{k,2}^{[m-1]},\ldots,u_{k,N}^{[m-1]})\\
    &=\left(-\frac{\partial}{\partial \eta_k}\rho(y,\boldeta)\big\rvert_{\boldeta=\hat{\boldeta}^{[m]}}\right)_{i=1}^N.
\end{align*}
Like in other gradient boosting algorithms, we need base learners.
In this algorithm, we specify component-wise base learners for each distribution parameter.
In principle, these might be different, but for simplicity, one usually chooses the same type for all, only letting the base learners differ in which component and which parameter they affect.
In other words, we have base learners
\begin{equation}
    h_{1,1},\ldots,h_{1,p_1},h_{2,1},\ldots,h_{2,p_2},\ldots,h_{K,p_K}.
\end{equation}
Based on the covariates and the residuals, we use the base learners to estimate possible updates of the model based on single predictors, and we choose the one that improves the model the most.

The initialization of the algorithm is done analogously to the regular boosting method, by setting each parameter to a constant, typically zero.
Alternatively, one might do a joint optimization of the log-likelihood, finding the optimal constant $c_k$ for each distribution parameter.
Then, initialize the estimate of each predictor $\eta_k$ as
\begin{equation}
    \hat{\eta}_k^{[0]}=\beta_{k,0},
\end{equation}
for each $k=1,2,\ldots,K$.

In the $k$-th step of iteration $m$, i.e. after having cycled through to component $k$, the estimated vector of additive predictors is denoted $\hat{\boldeta}_{k-1}^{[m]}$, and it is
\begin{equation}
    \hat{\boldeta}_{k-1}^{[m]}=\left(\hat{\eta}_1^{[m]},\hat{\eta}_2^{[m]},\ldots,\hat{\eta}_{k-1}^{[m]},\hat{\eta}_{k}^{[m-1]},\hat{\eta}_{k+1}^{[m-1]},\ldots,\hat{\eta}_{K}^{[m-1]}\right).
\end{equation}
Here the additive predictors of the preceding parameter dimensions, $1,2,\ldots,k-1$, have been updated in the current iteration $m$.
Hence they are denoted with the current iteration, as $\hat{\eta}_1^{[m]}$, $\hat{\eta}_2^{[m]}$, until $\hat{\eta}_{k-1}^{[m]}$.
The following dimensions $k+1,\ldots,K$ have not been updated in iteration $m$, and hence they are denoted with iteration $m-1$:
$\hat{\eta}_{k+1}^{[m-1]}$, $\hat{\eta}_{k+2}^{[m-1]}$, until $\hat{\eta}_{K}^{[m-1]}$.
We are in this step going to update dimension $k$, i.e., going from $hat{\eta}_{k}^{[m-1]}$ to $hat{\eta}_{k}^{[m]}$.
We calculate a residual for dimension $k$ by calculating the $k$-th partial derivative and inserting the current estimated vector of additive predictors $\hat{\boldeta}_{k-1}^{[m]}$, and evaluating it at the observations $x_1,x_2,\ldots,x_N$.
This yields the generalized residual vector
\begin{align*}
    \u_k^{[m-1]}&=(u_{k,1}^{[m-1]},u_{k,2}^{[m-1]},\ldots,u_{k,N}^{[m-1]})\\
    &=\left(-\frac{\partial}{\partial \eta_k}\rho(\hat{\eta}_1^{[m]},\hat{\eta}_2^{[m]},\ldots,\hat{\eta}_{k-1}^{[m]},\hat{\eta}_{k+1}^{[m-1]},\ldots,\hat{\eta}_{K}^{[m-1]})\right)_{i=1}^N \\
    &=\left(-\frac{\partial}{\partial \eta_k}\rho(y,\boldeta)\big\rvert_{\boldeta=\hat{\boldeta}_{k-1}^{[m]}}\right)_{i=1}^N.
\end{align*}
Again, like in a regular component-wise boosting algorithm, we fit all component-wise base learners separately to this residual vector $\u_k^{[m-1]}$.
Of these learners, select the best fitting component $j_k^{[m]}$ like previously, by selecting the estimated learner which fits best according to RSS,
\begin{equation}
    j_k^{[m]}=\argmin_{j\in\{1,2,\ldots,p_k\}}\sum_{i=1}^N \left(u_{k,i}^{[m-1]}-\hat{h}_{k,j}^{[m]}\right)^2.
\end{equation}
We update the additive predictor in dimension $k$ by the usual
\begin{equation}
    \hat{f}_k^{[m]}\gets\hat{f}_k^{[m-1]}+\nu\cdot \hat{h}^{[m]}_{j_k^{[m]}}(\cdot).
\end{equation}
A schematic representation of the updating process using this algorithm in a given iteration $m$ looks as follows:
\begin{align*}
    \frac{\partial}{\theta_1}\rho\left(y,\eta_1^{[m-1]},\eta_2^{[m-1]},\eta_3^{[m-1]},\ldots,\eta_{K-1}^{[m-1]},\eta_K^{[m-1]}\right)
    \xlongrightarrow{\text{calculate}}\u_{1}^{[m-1]}
    \xlongrightarrow{\text{fit and update}}\hat{f}_{1}^{[m]} \\
    \frac{\partial}{\eta_2}\rho\left(y,\hat{\eta}_1^{[m]},\hat{\eta}_2^{[m-1]},\hat{\eta}_3^{[m-1]},\ldots,\hat{\eta}_{K-1}^{[m-1]},\hat{\eta}_K^{[m-1]}\right)
    \xlongrightarrow{\text{calculate}} \u_{2}^{[m-1]}
    \xlongrightarrow{\text{fit and update}}\hat{f}_{2}^{[m]} \\
    \frac{\partial}{\eta_3}\rho\left(y,\hat{\eta}_1^{[m]},\hat{\eta}_2^{[m]},\hat{\eta}_3^{[m-1]},\ldots,\hat{\eta}_{K-1}^{[m-1]},\hat{\eta}_K^{[m-1]}\right)
    \xlongrightarrow{\text{calculate}}\u_{3}^{[m-1]}
    \xlongrightarrow{\text{fit and update}}\hat{f}_{3}^{[m]} \\
    \ldots\\
    \frac{\partial}{\eta_{K-1}}\rho\left(y,\hat{\eta}_1^{[m]},\hat{\eta}_2^{[m]},\hat{\eta}_3^{[m]},\ldots,\hat{\eta}_{K-1}^{[m-1]},\hat{\eta}_K^{[m-1]}\right)
    \xlongrightarrow{\text{calculate}}\u_{K-1}^{[m-1]}
    \xlongrightarrow{\text{fit and update}}\hat{f}_{K-1}^{[m]} \\
    \frac{\partial}{\eta_K}\rho\left(y,\hat{\eta}_1^{[m]},\hat{\eta}_2^{[m]},\hat{\eta}_3^{[m]},\ldots,\hat{\eta}_{K-1}^{[m]},\hat{\eta}_K^{[m-1]}\right)
    \xlongrightarrow{\text{calculate}}\u_{K}^{[m-1]}
    \xlongrightarrow{\text{fit and update}}\hat{f}_{K}^{[m]}
\end{align*}
For a schematic overview of this cyclical multidimensional boosting algorithm, see Algorithm \ref{algo:multi-cyclical}.
Note that this algorithm resembles the backfitting strategy of \citet{hastie1986}.

In both backfitting and this multidimensional boosting strategy, components are updated successively by using estimates of the other components as offset values.
In backfitting, a completely new estimate of $f^*$ is determined in every iteration.
In gradient boosting, however, the estimates are only slightly modified in each iteration.
The main tuning parameters in this algorithm are the stopping iterations $\boldsymbol{m}_{\text{stop}}=m_{\text{stop},1},\ldots,m_{\text{stop},K}$.
As in the one-dimensional gradient boosting algorithm, it should not run until convergence, but rather find estimates by cross-validation \citep{schmid}.
An issue with that, though, is that for proper tuning it requires a vector $\boldsymbol{m}_{\text{stop}}$ of stopping iterations, i.e., one stopping iteration for each prediction parameter.
To properly tune these parameters, it is necessary to perform a multidimensional search, which is usually done by implementing the so-called grid search.
One divides the search space into a multidimensional grid, obtaining tuples of configurations.
On each tuple, we should use cross-validation, as usual, and the next subsection explains the procedure.
\begin{algorithm}
\caption{Multidimensional cyclical component-wise gradient boosting}
\label{algo:multi-cyclical}
\begin{enumerate}
    \item Start with a data set $D=\{x_i, y_i\}_{i=1}^N$ and a chosen loss function $\rho(y,\boldeta)$, for which we wish to
        minimize the empirical risk, i.e., the loss function evaluated on the samples,
        \begin{equation}
            \hat{\boldeta}=\argmin_{\boldeta}R(\boldeta)=\argmin_{\boldeta}\sum_{i=1}^n \rho(y_i,\boldeta)\big\rvert_{\boldeta=\boldeta(x_i)}.
        \end{equation}
    \item Initialize iteration counter $m$ to 0. Initialize additive predictors $\beta_{0,1}^{[0]},\beta_{0,2}^{[0]},\ldots,\beta_{0,K}^{[0]}$. This can be done e.g., by setting these to zero, or by finding the best constants by maximizing the (joint) likelihood of the data.
    \item\label{initialization} Specify a set of base learners $\mathcal{H}_k$ for each predictor $\theta_k$, for $k=1,\ldots,K$. Specify a step length $\nu$.
    \item Increase $m$ by 1.
    \item Set $k$ to 0.
    \item\label{cyclic-first} Increase $k$ by 1.
    \item If $m>m_{\text{stop},k}$, go to step \ref{cyclic-first}. Otherwise compute the negative partial derivative
        $-\frac{\partial\rho}{\partial \eta_k}$ and evaluate at $\hat{\boldeta}^{[m-1]}(x_i),i=1,\ldots,N$, yielding the
        negative gradient vector
        \begin{equation}
            \u^{[m-1]}_k=\left(-\frac{\partial}{\partial \eta_k}\rho(y_i, \boldeta_i)\big\rvert_{\boldeta=\hat{\boldeta}_{i}^{[m-1]}}\right)_{i=1}^N
        \end{equation}
    \item Fit the negative gradient vector to each of the $p$ components of $\x$ separately, using each component's respective base learner. This yields $p$ vectors of predicted values,
        \begin{equation}
            \hat{h}_{k,1}^{[m]},\hat{h}_{k,2}^{[m]},\ldots,\hat{h}_{k,p}^{[m]}
        \end{equation}
        where each vector is an estimate of the negative gradient vector $\u^{[m-1]}_k$, or, again, a projection onto the space spanned by the component-wise learner.
    \item Select the component of $\x$ which fits best $\u^{(m-1)}_k$ (according to RSS), or, rather, the best-fitting base learner,
        \begin{equation}
            j_k^{[m]}=\argmin_{j\in\{1,2,\ldots,p\}}\sum_{i=1}^N \left(u_{k,i}-\hat{h}_{k,j}^{[m]}\right)^2.
        \end{equation}
    \item\label{cyclic-last} Update the predictor for parameter $k$ in component $j^{[m]}$ by
        \begin{equation}
            \hat{f}_{k,j_k^{[m]}}^{[m]}\gets\hat{f}_{k,j_k^{[m]}}^{[m-1]}+\nu\cdot\hat{h}_{j_k^{[m]}}^{[m]},
        \end{equation}
        where $\nu$ is the real-valued step-length factor specified in step \ref{initialization}. For all other components,
        meaning each $j\in\{j\neq j_k^{[m]},j=1,2,\ldots,p_k\},$ set the predictor to the one from the previous iteration,
        \begin{equation}
            \hat{f}_{k,j}^{[m]}\gets\hat{f}_{k,j}^{[m-1]}.
        \end{equation}
        Then update the full model thus far.
    \item Repeat steps \ref{cyclic-first} to \ref{cyclic-last} until $m=\max(m_{\text{stop},1},\ldots,m_{\text{stop},K})$.
    \item Return $\hat{\boldeta}$.
\end{enumerate}
\end{algorithm}

\subsection{Grid search cross-validation in gradient boosting}\label{grid-search}
To find a vector of length $K$ of optimal iterations $\boldsymbol{m}_{\text{stop}}=m_{\text{stop},1},\ldots,m_{\text{stop},K}$, we perform a $K$-dimensional grid search.
We must first specify a minimum and maximum number of iterations for each parameter.
Call these $m_{\min,k}$ and $m_{\max,k}$, respectively.
We then divide this one-dimensional search space into a finite grid with $N_k$ points, such that we obtain
\begin{equation} 
    m_{\min,k}=m_{1,k}<m_{2,k}<\ldots<m_{N_k-1,k}<m_{N_k,k}=m_{\max,k},
\end{equation}
again for each $k=1,2,\ldots,K$.
The total search space is the cartesian product of all of these grids.
We illustrate with an example. Let $K=3$, and $m_{\min,k}=1$ and $m_{\max,k}=10$ for all $k$, and finally divide each grid into 10 points, i.e., $N_1=N_2=N_3=10$.
The total search grid will consist of $N_1\cdot N_2\cdot N_3=10^3=10000$ tuples of configurations of $\boldsymbol{m}$, enumerated below:
\begin{align*}
    \left(m_{1,1},m_{1,2},m_{1,3}\right) \\
    \left(m_{1,1},m_{1,2},m_{2,3}\right) \\
    \ldots \\
    \left(m_{1,1},m_{1,2},m_{10,3}\right) \\
    \left(m_{1,1},m_{2,2},m_{1,3}\right) \\
    \ldots \\
    \left(m_{1,1},m_{2,2},m_{10,3}\right) \\
    \ldots \\
    \left(m_{10,1},m_{10,2},m_{10,3}\right).
\end{align*}
We want to find the best configuration $\boldsymbol{m}$, i.e., we want to find the optimum of the hyperplane $CV(\boldsymbol{m})$.
Like in subsection \ref{subsec:K-fold}, we must calculate the estimate of the cross-validated prediction error for each given configuration $\boldsymbol{m}$, obtaining the prediction error $CV(\boldsymbol{m})$.
We choose $\boldsymbol{m}_{\text{stop}}$ to be the minimizer of this error,
\begin{equation*}
    \boldsymbol{m}_{\text{stop}}=\argmin_{\boldsymbol{m}}\CV(\boldsymbol{m}).
\end{equation*}
Using boosting, we may obtain estimates of $\CV(\boldsymbol{m})$ for all $\CV(\boldsymbol{m})$ by fixing all but one of the parameters and perform a typical boosting run.
If we fix all but one of the parameters in the vector $\boldsymbol{m}=\left(m_{i_1,1},m_{i_2,2},m_{i_3,3}\right)$, where $m_{\min,k}\leq i_k\leq m_{\max,k}$ for all $k=1,2,3$, say, we fix $m_{i_1,1}$ and $m_{i_2,2}$.
This is due to the way boosting algorithms work, since for any given iteration $M$, we also automatically obtain all boosted estimates for all iterations less than $M$, if we have saved the boosted parameters for each iteration.
Consider again the example.
We now let the first two parameters in the example be fixed for each boosting run.
While the search grid consist of $N_1\cdot N_2\cdot N_3$ tuples, considering the first two parameters as fixed, we only need to do $N_1\cdot N_2$ boosting runs, and in each run set the maximum number of possible iterations in the boosting algorithm for the third component to be $m_{\max,3}$.
This means that we consider all configurations of the first two parameters in $\boldsymbol{m}$, i.e.,
\begin{align*}
    \left(m_{1,1},m_{1,2}\right) \\
    \left(m_{1,1},m_{2,2}\right) \\
    \ldots \\
    \left(m_{1,1},m_{10,2}\right) \\
    \left(m_{2,1},m_{1,2}\right) \\
    \ldots \\
    \left(m_{2,1},m_{10,2}\right) \\
    \ldots \\
    \left(m_{10,2},m_{10,2}\right),
\end{align*}
and do a boosting run for each such.
Like in subsection \ref{subsec:K-fold}, we choose
\begin{equation*}
    \boldsymbol{m}_{\text{stop}}=\argmin_{\boldsymbol{m}}\CV(\boldsymbol{m}),
\end{equation*}
where
\begin{equation*}
    \CV(\boldsymbol{m})=\sum_{k=1}^K\sum_{i\in\kappa(k)}\rho(y_i,\hat{y}_i^{-\kappa(k)}),
\end{equation*}
i.e., the cross-validated prediction error, as usual.
\section{Noncyclical component-wise multidimensional boosting algorithm}
In the cyclical algorithm seen previously in Algorithm \ref{algo:multi-cyclical}, the different $m_{\text{stop},j}$ parameters are not independent of each other, and hence they have to be jointly optimized.
As we saw in the previous subsection \ref{grid-search}, the usually applied \textit{grid search} for such parameters scales exponentially with the number of parameters $K$.
This can quickly become very demanding computationally.
\citet{thomas2018} develop a new algorithm for fitting GAMLSS models, instead of the cyclical one used in \textit{gamboostLSS}.
In this new algorithm, which they call ``noncyclical,'' only one scalar tuning parameter $m_{\text{stop}}$ is needed because only one parameter is chosen in each boosting iteration.
Compared to the cyclical algorithm in \textit{gamboostLSS} \citep{gamboostlss-paper}, this noncyclical algorithm obtains faster variable tuning and equal prediction results on simulation studies carried out \citep{thomas2018}.

\subsection{Gradients are not comparable across parameters}
In the cyclical algorithm, we always boost all parameters in the same iteration.
Therefore we do not need to choose between parameters.
If we want to avoid having a separate tuning parameter for each parameter that we are boosting, however, it is necessary to choose one parameter to boost in each iteration.
To do this we have to choose between parameters, and so we need to be able to find out which parameter would lead to the best increase in performance.
We already do this for choosing which component-wise learner to use in each parameter.
There, we choose that which has the best residual-sum-of-squares (RSS), with respect to the negative gradient vector.
\citet{thomas2018} denote this the \textit{inner loss}.
However, in general these generalized residual vectors are not comparable across parameters of the loss function, because the parameters have different scales \citep{thomas2018}.
In a normal distribution, for example, the partial derivatives for the mean and the partial derivative for the standard deviation will not be comparable.
Therefore, to compare between parameters, a different comparison method is needed.
We cannot compare the residual sums-of-squares, because they will not tell us which parameter will decrease the loss function the most.
For each parameter $\theta_k$, however, we choose the component-wise base learner which best fits according to the RSS,
\begin{equation}
    \hat{h}_{k,j}(\cdot),
\end{equation}
as usual.
If we incorporate this estimated base learner into the full boosted model, we would get
\begin{equation}
    \hat{\boldeta}^{[m]}_{k}=\hat{\boldeta}_k^{[m-1]}+\nu\cdot\hat{h}^{[m]}_{k,j}(\cdot).
\end{equation}
We can insert this proposed new model into the loss function to obtain a new empirical risk value,
\begin{equation}
    R\left(\hat{\boldeta}_{k}^{[m]}\right).
\end{equation}
We calculate the gain in the loss function, which we denote $\Delta\rho_{k}$, by
\begin{equation}
    \Delta\rho_{k}=R\left(\hat{\boldeta}^{[m-1]}\right)-R\left(\hat{\boldeta}_{k}^{[m]}\right),
\end{equation}
where $\hat{\boldeta}^{[m-1]}$, of course, is the current vector of additive predictors, from the previous iteration.
For each parameter $k=1,2,\ldots,K$, we can calculate its $\rho_{k}$.
If we now compare the gain in loss function value, we find out which parameter leads to the best increase. We choose that one, i.e.,
\begin{equation}
    k^{[m]}=\argmin_{k\in\{1,2,\ldots,K\}}\Delta\rho_{k}.
\end{equation}
We incorporate only the best-fitting learner corresponding for that parameter into the full boosting model, i.e.,
\begin{equation}
    \hat{\boldeta}^{[m]}\gets \hat{\boldeta}^{[m]}_{k}=\hat{\boldeta}^{[m-1]}+\nu\cdot\hat{h}^{[m]}_{k,j_k^{[m]}}(\cdot),
\end{equation}
meaning all components $k\in\{k\colon k\neq k^{[m]}, k=1,2,\ldots,K\}$ are not updated, so
\begin{equation}
    \hat{\eta}_{k}^{[m]}\gets\hat{\eta}_{k}^{[m-1]}.
\end{equation}

\subsection{Criterion for selecting component-wise learner}
We choose the base learner for each component by comparing their residual sum of squares with respect to the negative gradient vector, which we called the inner loss.
In other words, we use the usual procedure of choosing the component-wise learner for each parameter that minimizes the RSS, i.e.,
\begin{equation}
    j^{[m]}=\argmin_{j} \sum_{i=1}^N (u_{k,i}-(f^{[m]}(x_i)+\hat{h}(\x_i)))^2.
\end{equation}
This is, however, not the same criterion that is used to choose between parameters.
This might be problematic.
Therefore, \citet{thomas2018} propose using the loss function $\rho$ for choosing between component-wise learners as well.
They call this the ``outer loss.''
In the case of using the ``outer loss,'' in each parameter-specific step, we choose instead the component-wise learner that minimizes the outer loss function, i.e., 
\begin{equation}
    j^{[m]}=\argmin_{j} \sum_{i=1}^N \rho(y_i,f(x_i)).
\end{equation}
The individual component-wise learners are still estimated by their usual method, i.e., calculating the negative gradient of the generalized residuals and using the base learner to estimate models.
E.g., by linear least squares if using simple linear regression base learners.
The improvement in the empirical risk, $\Delta\rho_{k}$, is then calculated for each base learner of every distribution parameter, and only the overall best-performing base learner with regard to the outer loss is updated.

In both cases of this algorithm, we have the advantage that the optimal number of boosting steps, $m_{\text{stop}}$, is always a scalar value.
Finding this tuning parameter can be done fairly quickly with standard cross validation schemes, and most importantly, it scales with with the number of parameters.

A schematic overview of the algorithm is given in algorithm \ref{algo:multi-noncyclical}.

\begin{algorithm}
\caption{Multidimensional noncyclical component-wise gradient boosting}
\label{algo:multi-noncyclical}
\begin{enumerate}
    \item Start with a data set $D=\{x_i, y_i\}_{i=1}^N$ and a chosen loss function $\rho(y,\hat{f}(x))$, for which we wish to
        minimize the empirical risk, i.e., the loss function evaluated on the samples,
        \begin{equation}
            \hat{f}=\argmin_{f}R(f).
        \end{equation}
    \item Set $m=0$. Initialize $f^{(0)}_1,f^{(0)}_2,\ldots,f^{(0)}_K,$, e.g., by setting it to zero for all components, or by finding the best constant.% Initialize scale (nuisance) parametes $\sigma$.
    \item Specify a base learner $h_k$ for each dimension $k=1,\ldots,K$.
    \item Increase $m$ by 1.
    \item Set $k=0$.
    \item Increase $k$ by 1.
    \item Compute the negative partial derivative $-\frac{\partial\rho}{\partial \eta_k}$
        and evaluate at $\hat{\boldeta}^{[m-1]}(x_i),i=1,\ldots,N$, yielding negative gradient vector
        \begin{equation}
            \u^{[m-1]}_k=\left(-\frac{\partial}{\partial \hat{\eta}_k}\rho(y_i, \hat{\boldeta}^{[m-1]}(x_i))\right)_{i=1}^N
        \end{equation}
    \item Fit the negative gradient vector to each of the $p$ components of $X$ (i.e. to each base learner) separately, using the base learners specified in step X. This yields $p$ vectors of predicted values, where each vector is an estimate of the negative gradient vector $\u^{[m-1]}_k$.
    \item Select the best fitting base learner, $h_{kj}$, either by
        \begin{itemize}
            \item the inner loss, i.e., the RSS of the base-learner fit w.r.t the negative gradient vector
                \begin{equation}
                    j^*=\argmin_{j\in 1,\ldots,J_k}\sum_{i=1}^N(u_k^{(i)}-\hat{h}_{kj}(x^{(i)}))^2
                \end{equation}
            \item the outer loss, i.e., the loss function after the potential update,
                \begin{equation}
                    j^*=\argmin_{j\in 1,\ldots,J_k}\sum_{i=1}^N\rho\left(y^{(i)}, \hat{f}^{(m-1)}(x^{(i)}) + \nu \cdot \hat{h}_{kj}(x^{(i)}) \right)
                \end{equation}
        \end{itemize}
    \item Compute the possible improvement of this update regarding the outer loss,
        \begin{equation}
            \Delta\rho_k=\sum_{i=1}^N\rho\left(y^{(i)}, \hat{\boldeta}^{[m-1]}(x^{(i)}) + \nu \cdot \hat{h}_{kj^*}(x^{(i)}) \right)
        \end{equation}
    \item Update, depending on the value of the loss reduction, $k^*=\argmin_{k\in1,\ldots,K}\Delta\rho_k$
        \begin{equation}
            \hat{\eta}^{[m]}_{k^*}=\hat{\eta}^{[m-1]}_{k^*}+\nu\cdot\hat{h}_{k^*j^*}(x),
        \end{equation}
        while for all $k\neq k^*$,
        \begin{equation}
            \hat{\eta}^{[m]}_{k^*}=\hat{\eta}^{[m-1]}_{k^*}.
        \end{equation}
    \item Repeat steps 4 to 11 until $m=m_{\text{stop}}$.
    \item Return $\hat{\boldeta}(\cdot)=\hat{\boldeta}_M(\cdot)=\sum_{m=0}^M\boldeta_m(\cdot)$.
\end{enumerate}
\end{algorithm}


%\section{Centering and scaling of covariates is important to ensure proper fitting}
%\subsection{Centering}
%We are using component-wise linear least squares base learners \textit{without} intercepts, i.e., functions of the form
%\begin{equation*}
%    h_j(\x_i)=\beta_jx_{i,j}.
%\end{equation*}
%Here $i$ is an observation from the observed data set, i.e., $i\in\{1,2,\ldots,n\}$, and $j$ is a column of the covariate matrix, i.e., $j\in\{1,2\ldots,p\}$.
%Let $\X$ be a stochastic vector which is randomly drawn from the data distribution, and let $X_j$ be the $j$-th component of this vector.
%If column $j$ is centered, this means that
%\begin{equation*}
%    \mathbb{E}[X_j]=0.
%\end{equation*}
%Hence it follows that
%\begin{equation*}
%    \mathbb{E}[h_j(\X)]=\mathbb{E}[\beta_jX_{j}]=\mathbb{E}[\beta_j]\mathbb{E}[X_j]=\mathbb{E}[\beta_j]\cdot0=0.
%\end{equation*}
%This means that the interpretation of the estimated $\hat{\beta}_j$ is ``the effect of a unit increase in $x_j$ over the expected value of $x_j$''.
%If, however, the column is not centered, then
%\begin{equation*}
%    \mathbb{E}[X_j]=x_j^*,
%\end{equation*}
%Hence it follows that
%\begin{equation*}
%    \mathbb{E}[h_j(\X)]=\mathbb{E}[\beta_jX_{j}]=\mathbb{E}[\beta_j]\mathbb{E}[X_j]=\mathbb{E}[\beta_j]x_j^*.
%\end{equation*}